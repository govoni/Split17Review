\label{WG2}

\subsection{Experimental Overview -- N. Lorenzo Martinez}
At the time of the workshop, a number of experimental results in VBS have been available, all of them from the LHC experiments CMS and ATLAS (see Tab.~\ref{tab:wg2:expres}). The highlight among these results is a measurement from the CMS experiment in the $W^\pm W^\pm$ channel, which for the first time shows the existence of the electroweak contribution in VBS processes at high significance (5 $\sigma$)~\cite{CMS:2017adb}.

By now, a large fraction of the different possible final state boson combinations have been studied by at least one experiment, with the notable absence of the $\gamma\gamma$ and $W^\pm W\mp$ channels. This wide coverage of channels has been shown to be very helpful when constraining aQGCs, as the different channels show varying sensitivity to different operators.
\textbf{add some figure here?} Even though, progress has been made, it would still be advantegeous if both experiments could cover all relevant channels.

\begin{table}[htb]
\centering
\label{tab:wg2:expres}
\title{Experimental results on VBS processes by final state. Only the most recent result from each experiment is shown.}
\begin{tabular}{|l|c|c|}
    \hline
    channel & ATLAS & CMS \\
    \hline
    $Z(\ell\ell)\gamma$ & \cite{Aaboud:2017pds} & \cite{Khachatryan:2017jub} \\
    $Z(\nu\nu)\gamma$ &  \cite{Aaboud:2017pds}& $\times$ \\
    $W^\pm W^\pm$ & \cite{Aaboud:2016ffv} & \cite{CMS:2017adb} \\
    $W(\ell\nu)\gamma$ & $\times$ & \cite{Khachatryan:2016vif} \\
    $Z(\ell\ell)Z(\ell\ell)$&  $\times$  & \cite{CMS-PAS-SMP-17-006} \\
    $W(\ell\nu)Z(\ell\ell)$ & \cite{Aad:2016ett} & $\times$ \\
    $W(\ell\nu)V(qq)$ & \cite{Aaboud:2016uuk} & $\times$ \\
    \hline
  \end{tabular}  
\end{table}

Although the avialability of such a wide range of results bodes well for future progress, the presentation and interpretation of results in the presented studies shows some notable differences.

Different approaches are used to treat unitarity issues that can arise when the analysis sensitivity to potential aQGCs is not high enough to exclude aQGC values small enough to guarnatee unitarity within the LHCs energy scale. Chosing different approaches here severly complicates the combination of different measurements of limits on aQGCs, even though such combinations could substantially increase the statistical power of the total dataset as well as break degeneracies between the effects of different operators which may affect a single channel in similar ways.

Different approaches are chosen when treating the interference effects between electroweak and QCD amplitudes in the predictions for SM cross sections. Again, such differences complicate the combination of results, though for this effect the combination of SM cross section measurements is affected. At the current level of experimental accuracy, the differences in treatement of interference effects are still small compared to experimental accuracy, but with the continued data-taking at the LHC a common approach would be desireable. 

Providing recommendations to resolve the above two issues should be an important goal for WG2. {\bf This doesn't sound so nice, but how to improve}

\subsection{Common Selection Criteria -- X. Janssen}

In order to facilitate feasibility studies and similar forward looking analysis in a way that allows for a fair comparison between such studies, it will be useful to define a common baseline for a fiducial phasespace. However, a single definition cannot serve as the base for every study for the following reasons:
\begin{itemize}
\item The experiments themselves will evolve due to hardware upgrades planned for the high luminostiy phase of the LHC, necessitating different assumptions on the detector acceptance depending on the integrated luminosity used for the forward lookgin study.
\item Different final VBS channels and different boson decay modes may require notably different selection criteria.
\end{itemize}

The simplest case is realized for near term studies of VBS channels with signals large enough to be amenable to simple ``cut \& count'' style analysis, notably the $W^\pm W^\pm$ channel. Based on published results, a phase-space region close to CMS as well as ATLAS studies has been identified (see Tab.~\ref{tab:wg2:phasespac}). This is likely to be useful for forward looking studies reaching up to, but excluding the high luminostiy phase of the LHC.

Further extrapolation into the future suffers from problem, that the design work for the envisioned detector upgrades is not entirely completed yet, so that work in this area is necessarily somewhat speculative. Nevertheless a few general conclusions can be drawn from existing detector design documents. Both experiments plan to improve their trigger systems to keep the thresholds for lepton triggers at a similar level to current running conditions, so that $p_{\mathrm{T}}$ thresholds should remain close to current ones. Both expseriments also plan to extend coverage of their repsective tracking detectors up to $|\eta|\sim 4$. However due to differences in the other systems relevant for lepton identification (electromagnetic calorimeter and muons systems), the final geometric coverage for the leptons is not yet certain and may differ substantially between experiments.

While the common phase space definitions discussed here are useful for the study of channels that are experimentally accessible with straight forward techniques, they are less suitable for studies of channels with very low cross section, branching fraction or efficiency, as exemplified by the $ZZ\rightarrow 4\ell$ channel. Due to the low number of observable events in this channel, the existing analyis~\cite{CMS-PAS-SMP-17-006} is performed in a very inclusive phase space, employing sophisticated mutli-variate techniques to isolate the signal. Such multivariate techniques are difficult to model in forward looking studies in a simple manner, and an ad-hoc implementation of a  ``cut \& count'' style study is likely to severly underestimate the future performace of the LHC detectors. For this case, the scaling of existing results according to the expected growth with luminosity of signal and background yields, will be preferable.

\begin{table}[htb]
\centering
\label{tab:wg2:phasespace}
\title{Proposed phase space for studies in the WW channel.}
\begin{tabular}{|l|l|l|l|l|}
    \hline
             & electrons & muons & jets & photons \\
    \hline
    $|\eta|$ & $<2.5$  & $<2.4$ & $<4.5$ & $<2.5$ \\
    $p_\mathrm{T,lead.}$ & $>25$~GeV & $>25$~GeV &$>30$~GeV &$>25$~GeV\\
    $p_\mathrm{T,sublead.}$ & $>15$~GeV & $>15$~GeV &&\\                            
    \hline
  \end{tabular}  
\end{table}

\subsection{Prospects -- M. Kobel}

Major topics:
Longitudinal component.
purest channel (WW), polarization is not accessible. Channels with easy pol. typically have high QCD contributions.
Some ideas to improve by devising polarization depndent variables.

BSM interpretations.
Advantages / Disadvantages of EFTs vs more explicit resonance modeling.
EFT advantage: generic, complete, computations avaliable.
EFT disadvantage: unitarization issues. Poor modeling of resonance tails (i.e. the ``$<<$'' in the usual assumption of $<<\Lambda$ in EFTs can be ver large)

Need some figures here?
