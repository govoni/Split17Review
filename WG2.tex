\label{WG2}

\subsection{Experimental Overview -- N. Lorenzo Martinez}
At the time of the workshop, a number of experimental results in VBS have been available, all of them from the LHC experiments CMS and ATLAS (see Tab.~\ref{tab:wg1:expres}). The highlight among these results is a measurement from the CMS experiment in the $W^\pm W^\pm$ channel, which for the first time shows the existence of the electroweak contribution in VBS processes at high significance (5 $\sigma$)~\cite{CMS:2017adb}.

By now, a large fraction of the different possible final state boson combinations have been studied by at least one experiment, with the notable absence of the $\gamma\gamma$ and $W^\pm W\mp$ channels. This wide coverage of channels has been shown to be very helpful when constraining aQGCs, as the different channels show varying sensitivity to different operators.
\textbf{add some figure here?} Even though, progress has been made, it would still be advantegeous if both experiments could cover all relevant channels.

\begin{table}[htb]
\centering
\label{tab:wg1:expres}
\title{Experimental results on VBS processes by final state. Only the most recent result from each experiment is shown.}
\begin{tabular}{|l|c|c|}
    \hline
    channel & ATLAS & CMS \\
    \hline
    $Z(\ell\ell)\gamma$ & \cite{Aaboud:2017pds} & \cite{Khachatryan:2017jub} \\
    $Z(\nu\nu)\gamma$ &  \cite{Aaboud:2017pds}& $\times$ \\
    $W^\pm W^\pm$ & \cite{Aaboud:2016ffv} & \cite{CMS:2017adb} \\
    $W(\ell\nu)\gamma$ & $\times$ & \cite{Khachatryan:2016vif} \\
    $Z(\ell\ell)Z(\ell\ell)$&  $\times$  & \cite{CMS-PAS-SMP-17-006} \\
    $W(\ell\nu)Z(\ell\ell)$ & \cite{Aad:2016ett} & $\times$ \\
    $W(\ell\nu)V(qq)$ & \cite{Aaboud:2016uuk} & $\times$ \\
    \hline
  \end{tabular}  
\end{table}

Although the avialability of such a wide range of results bodes well for future progress, the presentation and interpretation of results in the presented studies shows some notable differences.

Different approaches are used to treat unitarity issues that can arise when the analysis sensitivity to potential aQGCs is not high enough to exclude aQGC values small enough to guarnatee unitarity within the LHCs energy scale. Chosing different approaches here severly complicates the combination of different measurements of limits on aQGCs, even though such combinations could substantially increase the statistical power of the total dataset as well as break degeneracies between the effects of different operators which may affect a single channel in similar ways.

Different approaches are chosen when treating the interference effects between electroweak and QCD amplitudes in the predictions for SM cross sections. Again, such differences complicate the combination of results, though for this effect the combination of SM cross section measurements is affected. At the current level of experimental accuracy, the differences in treatement of interference effects are still small compared to experimental accuracy, but with the continued data-taking at the LHC a common approach would be desireable. 

Providing recommendations to resolve the above two issues should be an important goal for WG2. {\bf This doesn't sound so nice, but how to improve}

\subsection{Common Selection Criteria -- X. Janssen}

Goal have realistic phase-space definition for theory- and feasability studies

Comparison of selections of existing analysis.
Reasonable baseline possible for ``easy'' channels (e.g. WW), where signal is accessible in cut \& count style analysis.

Ultimately difficult for signals with very low cross-section / BR / efficiency (e.g. ZZ), as these require very loose phase-space selection + sophisticated MVA final selection, which is not easily modeled for the above purpose.

\subsection{Prospects -- M. Kobel}

Major topics:
Longitudinal component.
purest channel (WW), polarization is not accessible. Channels with easy pol. typically have high QCD contributions.
Some ideas to improve by devising polarization depndent variables.

BSM interpretations.
Advantages / Disadvantages of EFTs vs more explicit resonance modeling.
EFT advantage: generic, complete, computations avaliable.
EFT disadvantage: unitarization issues. Poor modeling of resonance tails (i.e. the ``$<<$'' in the usual assumption of $<<\Lambda$ in EFTs can be ver large)

Need some figures here?
