
In the past years, the study of the VBS processes has attracted lots of interests from the theory community, (see {\emph e.g.} the review~\cite{Rauch:2016pai}).
The acivities range from providing precise predictions for the VBS signal and background processes in the Standard Model, to studying the sensitivity of these 
processes on new physics effects, by mean of effective theories or by studying the polarisation patterns of the heavy gauge bosons involved.
The program of the meeting reflects these avenues.

\subsection{Complete NLO corrections to ${\rm W^+ W^+}$ scattering - M. Pellen}

The first VBS process that has been observed during the run~I of the LHC is the same-sign WW production~\cite{Aad:2014zda,Aaboud:2016ffv,Khachatryan:2014sta}.
This observation has already been confirmed by a measurement of the CMS collaboration at the 13 TeV run~II~\cite{CMS:2017adb}.
In view of the growing mole of data which will be collected by the experiments, and of the consequent reduction of the uncertainties affecting these measurements, precise theoretical predictions become necessary.

In that respect NLO QCD and EW corrections to such signature should be computed.
So far, NLO have focused on NLO QCD corrections to the VBS process~\cite{Jager:2009xx,Jager:2011ms,Denner:2012dz,Rauch:2016pai} and its QCD-induced irreducible background process~\cite{Melia:2010bm,Melia:2011gk,Campanario:2013gea,Baglio:2014uba,Rauch:2016pai}.
No NLO EW have been computed and the NLO QCD corrections relied on the so-called VBS approximation.
In Refs.~\cite{Biedermann:2016yds,Biedermann:2017bss}, for the first, all leading order (LO) and next-to-leading (NLO) contributions to the full ${\rm p}{\rm p}\to\mu^+\nu_\mu{\rm e}^+\nu_{\rm e}{\rm j}{\rm j}$ process have been reported.
As the full amplitudes are used, this amounts to compute three LO contributions and four NLO contributions.
At LO, the three contributions are the EW process (order $\mathcal{O}{\left(\alpha^{6}\right)}$), its QCD-induced counterpart (order $\mathcal{O}{\left(\alpha_{\rm s}\alpha^{5}\right)}$) as well as the interference (order $\mathcal{O}{\left(\alpha_{\rm s}^2\alpha^{4}\right)}$).
Due to the VBS event selection applied to the final state, the full process is dominated by the purely EW contribution (see Table~\ref{table:LOVBS}).
This EW contribution feature the VBS diagrams but also background diagrams where for example the W bosons are simply radiated from the quark lines.

\begin{table}
\begin{center}
\begin{tabular}{|l||c|c|c||c|}
\hline
Order & $\mathcal{O}{\left(\alpha^{6}\right)}$ & $\mathcal{O}{\left(\alpha_{\rm s}\alpha^{5}\right)}$ & $\mathcal{O}{\left(\alpha_{\rm s}^2\alpha^{4}\right)}$ & Sum \\
\hline
\hline
${\sigma_{\mathrm{LO}}}$ [fb] 
& $1.4178(2)$
& $0.04815(2)$
& $0.17229(5)$
& $1.6383(2)$ \\
\hline
\end{tabular}
\end{center}
\caption{
Fiducial cross section~\cite{Biedermann:2017bss} at LO for the process ${\rm p}{\rm p}\to\mu^+\nu_\mu{\rm e}^+\nu_{\rm e}{\rm j}{\rm j}$, at
orders  $\mathcal{O}{\left(\alpha^{6}\right)}$, $\mathcal{O}{\left(\alpha_{\rm s}\alpha^{5}\right)}$, and $\mathcal{O}{\left(\alpha_{\rm s}^2\alpha^{4}\right)}$.
The sum of all the LO contributions is in the last column and all contributions expressed in femtobarn. 
The statistical uncertainty from the Monte Carlo integration on the last digit is given in parenthesis.}
\label{table:LOVBS}
\end{table}

At NLO, the four contributions arise at the orders $\mathcal{O}{\left(\alpha^{7}\right)}$, $\mathcal{O}{\left(\alpha_{\rm s}\alpha^{6}\right)}$, $\mathcal{O}{\left(\alpha_{\rm s}^{2}\alpha^{5}\right)}$, and $\mathcal{O}{\left(\alpha_{\rm s}^{3}\alpha^{4}\right)}$. 
An interesting feature is that the orders $\mathcal{O}{\left(\alpha_{\rm s}\alpha^{6}\right)}$ and $\mathcal{O}{\left(\alpha_{\rm s}^{2}\alpha^{5}\right)}$ receive both EW and QCD corrections.
Thus, at NLO (as opposed to LO) it is not possible to strictly distinguish the EW process from the QCD-induced process.
As it can be seen from Table~\ref{table:NLOVBS}, at the level of the fiducial cross section, the largest corrections are the one of order $\mathcal{O}{\left(\alpha^{7}\right)}$.
These are the NLO EW corrections to the EW processes.

\begin{table}
\begin{center}
\begin{tabular}{|l||c|c|c|c||c|}
\hline
Order & $\mathcal{O}{\left(\alpha^{7}\right)}$ & $\mathcal{O}{\left(\alpha_{\rm s}\alpha^{6}\right)}$ & $\mathcal{O}{\left(\alpha_{\rm s}^{2}\alpha^{5}\right)}$ & $\mathcal{O}{\left(\alpha_{\rm s}^{3}\alpha^{4}\right)}$ & Sum \\
\hline
\hline 
${\delta \sigma_{\mathrm{NLO}}}$ [fb] 
& $-0.2169(3)$ 
& $-0.0568(5)$
& $-0.00032(13)$
& $-0.0063(4)$ 
& $-0.2804(7)$ \\
\hline
$\delta \sigma_{\mathrm{NLO}}/\sigma_{\rm LO}$ [\%] & $-13.2$ & $-3.5$ & $0.0$ & $-0.4$ & $-17.1$ \\
\hline
\end{tabular}
\end{center}
\caption{
NLO corrections~\cite{Biedermann:2017bss} for the process ${\rm p}{\rm p}\to\mu^+\nu_\mu{\rm e}^+\nu_{\rm e}{\rm j}{\rm j}$ at the orders 
$\mathcal{O}{\left(\alpha^{7}\right)}$, $\mathcal{O}{\left(\alpha_{\rm s}\alpha^{6}\right)}$, $\mathcal{O}{\left(\alpha_{\rm s}^{2}\alpha^{5}\right)}$, and $\mathcal{O}{\left(\alpha_{\rm s}^{3}\alpha^{4}\right)}$.
The sum of all the NLO contributions is in the last column.
The contribution $\delta\sigma_{\mathrm{NLO}}$ corresponds to the absolute correction while $\delta \sigma_{\mathrm{NLO}}/\sigma_{\rm LO}$ gives the relative correction normalised to the sum of all LO contributions.
The absolute contributions are expressed in femtobarn while the relative ones are expressed in per cent.
The statistical uncertainty from the Monte Carlo integration on the last digit is given in parenthesis.}
\label{table:NLOVBS}
\end{table}

All these different contributions can also be seen in two differential distributions in the the transverse momentum for the hardest jet and invariant mass for the two leading jets in Fig.~\ref{fig:VBSALL}.
Each contribution (both LO and NLO) shows distinctive features.
At LO, the QCD induced process as well as the interference are rather suppressed due to the typical VBS event selection (as for the fiducial cross section).
This is exemplified in the invariant mass of the two leading jet where at high invariant mass, the QCD-induced background becomes negligible.
At NLO, the bulk of the corrections originate from the large EW to the EW process at order $\mathcal{O}{\left(\alpha^{7}\right)}$.
In particular they display the typical behaviour of Sudakov logarithms that grow large in the high energy limit.
Note that the contribution from initial state photon is also shown but is not taken into account in the definition of the NLO predictions.
This contribution is rather small and fairly constant in shape over the whole range studied here.
Finally, as at NLO, in it is not possible to isolate the EW production from its irreducible backgrounds, a global measurement of the process full ${\rm p}{\rm p}\to\mu^+\nu_\mu{\rm e}^+\nu_{\rm e}{\rm j}{\rm j}$ with all component is desirable.
Hence in Fig.~\ref{fig:VBSAAL_sum}, the sum all the contributions at both LO and NLO is shown in a combined prediction.

\begin{figure}
% \hspace{-2cm}
\includegraphics[width=.47\textwidth]{WG1_plots/histogram_transverse_momentum_j1}
\hfill
\includegraphics[width=.47\textwidth]{WG1_plots/histogram_invariant_mass_mjj12}
% \vspace*{-1em}
\caption{Differential distributions~\cite{Biedermann:2017bss} for ${\rm p}{\rm p}\to\mu^+\nu_\mu{\rm e}^+\nu_{\rm e}{\rm j}{\rm j}$:
transverse momentum for the hardest jet~(left) and invariant mass for the two leading jets~(right).
The two lower panels show the relative NLO corrections with respect to the full LO in per cent,
defined as $\delta_i = \delta \sigma_{i} / \sum \sigma_{\text{LO}}$, 
where $i=\mathcal{O}{\left(\alpha^{7}\right)},\mathcal{O}{\left(\alpha_{\rm s}\alpha^{6}\right)},\mathcal{O}{\left(\alpha_{\rm s}^2\alpha^{5}\right)},\mathcal{O}{\left(\alpha_{\rm s}^3\alpha^{4}\right)}$.
In addition, the NLO photon-induced contributions of order $\mathcal{O}{\left(\alpha^{7}\right)}$ is provided separately.}
\label{fig:VBSALL}
\end{figure}

\begin{figure}
% \hspace{-2cm}
\includegraphics[width=.47\textwidth]{WG1_plots/histogram_invariant_mass_epmu_scale}
\hfill
\includegraphics[width=.47\textwidth]{WG1_plots/histogram_rapidity_separation_pomu_scale}
% \vspace*{-1em}
\caption{Differential distributions~\cite{Biedermann:2017bss} for ${\rm p}{\rm p}\to\mu^+\nu_\mu{\rm e}^+\nu_{\rm e}{\rm j}{\rm j}$:
invariant mass of the positron and anti-muon system~(left) and rapidity separation between the electron and anti-muon~(right).
The upper panels show the sum of all LO and NLO contributions with scale variation.
The lower panels show the relative corrections in per cent.}
\label{fig:VBSAAL_sum}
\end{figure}

As shown previously, the EW corrections of order $\mathcal{O}\left(\alpha^7\right)$ are the dominant NLO contributions.
These originate from Sudakov logarithms that grow negatively large in the high energy limit.
This is shown in the differential distribution in the transverse momentum of the hardest jet on the left hand-side of Fig.~\ref{fig:VBSEW}.
Usually these EW corrections are particularly large only in phase space regions which are suppressed.
Hence the impact at the level of the total fiducial cross section is usually rather limited.
This is not the case here where the corrections are already large at the level of the cross section and reach $-16\%$ \cite{Biedermann:2016yds}.
The origin of these large EW corrections are indeed virtual corrections and in particular the ones corresponding to the insertion of massive particles in the scattering process \cite{Biedermann:2016yds}.
Hence, large NLO EW corrections are an intrinsic feature of VBS at the LHC.
As the EW corrections are particularly large, it might be possible to measure them at a high luminosity LHC, hence probing the EW sector of the Standard Model to very high precision.
This is illustrated on the left hand-side of Fig.~\ref{fig:VBSEW} where the band represent the estimated statistical error for a high-luminosity LHC collecting $3000{\rm fb}^{-1}$.

\begin{figure}
% \hspace{-2cm}
\includegraphics[width=.47\textwidth]{WG1_plots/histogram_transverse_momentum_j1_ew}
\includegraphics[width=.47\textwidth]{WG1_plots/histogram_rapidity_j1j2_ew}
% \vspace*{-1em}
\caption{Differential distributions~\cite{Biedermann:2016yds} for ${\rm p}{\rm p}\to\mu^+\nu_\mu{\rm e}^+\nu_{\rm e}{\rm j}{\rm j}$ including NLO EW corrections (upper panel) and relative NLO EW corrections (lower panel).
Left plot: Invariant-mass distribution of the four leptons.
Right plot: Rapidity distribution of the leading jet pair.
The yellow band describes the expected statistical experimental uncertainty for a high-luminosity LHC collecting $3000{\rm fb}^{-1}$ and represents a relative variation of $\pm 1/\sqrt{N_{\rm obs}}$ where $N_{\rm obs}$ is the number of observed events in each bin.}
\label{fig:VBSEW}
\end{figure}

\subsection{Monte Carlo comparisons for ${\rm W^+ W^+}$ scattering - M. Zaro}
In the last decade many codes capable of performing VBS simulations have appeared; within a network such as VBSCan, it is therefore natural to perform 
a quantitative comparison of these codes, both to cross-validate the results and to assess the impact of the different approximations which are performed in each tool. In fact, 
already at the LO, when the process ${\rm p}{\rm p}\to\mu^+\nu_\mu{\rm e}^+\nu_{\rm e}{\rm j}{\rm j}$ at $\mathcal O (\alpha^6)$ is considered, the various implementations of VBS differ, for example
by the (non-)inclusion of diagrams with vector bosons in the $s$-channel or by the treatment of interferences between diagrams; the reason of these differences is that, 
when typical signal cuts for VBS are imposed, these effects turn to be totally negligible on rates and distributions.

In our comparison, we use the following codes: {\bf ADD/CHECK CITES} 
{\sc Bonsay}~\cite,
{\sc MadGraph5\_aMC@NLO}~\cite{Alwall:2014hca}, 
{\sc Powheg}~\cite{},
{\sc Recola+MoCaNLO}~\cite{Actis:2012qn,Actis:2016mpe,MoCaNLO}\footnote{To numerically evaluate the one-loop scalar and tensor integrals, {\sc Recola} relies on the {\sc Collier} library \cite{Denner:2014gla,Denner:2016kdg}},
{\sc VBFNLO}~\cite{Arnold:2008rz, Arnold:2011wj, Baglio:2014uba}.
The complete comparison of the codes will be published in a separate work. Here, we present some preliminary results obtained at LO ($\mathcal O (\alpha^6)$) and including
NLO QCD corrections at fixed-order ($\mathcal O (\alpha^6\alpha_s)$, for the process ${\rm p}{\rm p}\to\mu^+\nu_\mu{\rm e}^+\nu_{\rm e}{\rm j}{\rm j}$. In 
Tab.~\ref{tab:wg1_codes} we report on the details of the various codes. In particular, we check whether
\begin{itemize}
    \item all $s-$ and $t/u-$channel diagrams that lead to the considered final state are included;
    \item interferences between diagrams are included at LO;
    \item off-shell contributions and diagrams which do not feature two resonant vector bosons are included;
    \item the so-called ``non-factorizable'' (NF) QCD corrections, that is those corrections where (real or virtual) gluons are exchanged between different quark lines,
        are included;
    \item EW corrections to the $\mathcal O (\alpha^5\alpha_s)$ interference are included. These corrections are of the same order as the NLO QCD corrections to
        the  $\mathcal O (\alpha^6$ term.
\end{itemize}
%
\begin{table}
    \footnotesize
    \begin{tabularx}{\textwidth}{c|c|X|X|X|X|X}
        Contact person  &  Code  &  $\mathcal O(\alpha^6)$ $|s|^2/$ $|t|^2/|u|^2$  &  $\mathcal O(\alpha^6)$ interf.  &  Off-shell  &  NF QCD  &  EW corr. to $\mathcal O(\alpha^5\alpha_s)$  \\
        \hline
        \hline
        A. Karlberg  &  {\sc POWHEG}  &  $t/u$  &  No  &  Yes  &  No  &  No  \\
        M. Pellen    &  {\sc Recola}  &  Yes  &  Yes  &  Yes  &  Yes  &  Yes  \\
        M. Rauch     &  {\sc VBFNLO}  &  Yes  &  No  &  Yes  &  No  &  No  \\
        C. Schwan    &  {\sc BONSAY}  &  $t/u$  &  No  &  Yes, virt. No  &  No  &  No  \\
        M. Zaro      &  {\sc MG5\_aMC}  &  Yes  &  Yes  &  No virt.  &  No  &  No
    \end{tabularx}
    \caption{\label{tab:wg1_codes} Summary of the different properties of the codes employed in the comparison.}
\end{table}
%
INPUT AND SETUP\\
VBS CUTS\\
PLOTS\\
The results of our comparison show an excellent agreement of the different tools at LO, which justify the approximations made in some of these tools not to include certain 
contributions. At NLO, slightly larger (but still below $10\%$) effects can be noticed for example at low dijet invariant mass, where $s-$ channels and interferences are not
as much suppressed as at LO, because of real QCD radiation.


We conclude this session by recalling that the results presented must be regarded as preliminary.
In the coming months, this work will be enlarged to include comparison of predictions at NLO QCD matched to parton shower or with EW corrections, 
as well as to study the effect of changing 
the VBS cuts. The QCD-induced background will also be studied.

\subsection{Polarisation of vector bosons - E. Maina}

It has been argued that new physics model could lead to large contributions to the longitudinal polarisation of heavy gauge bosons [citation?].
Hence it is of prime importance to study the polarisation of the heavy gauge bosons and their effects both in the Standard Model and beyond.
In particular, it is very important to devise methods that allow for such studies.
Thus, a starting point is to establish these methods first for the Standard Model.
In this respect, a new method has been proposed and applied to the scattering of two W bosons of opposite charge.

First, a matrix element squared can be written in a general manner as

\begin{equation}
\label{eq:polarisation}
 \mathcal{A}^2 = \sum_{\lambda} \mathcal{A}^2_{\lambda, \lambda} + \sum_{\lambda \neq \lambda'} \mathcal{A}^2_{\lambda, \lambda'}, 
\end{equation}
%
where $\lambda$ represents the polarisation of the heavy gauge bosons.
Usually the last term of Eq.~\eqref{eq:polarisation} is assumed to be zero.
But this is only true when ones integrates over the full phase space over the variable $\phi$.
In experiments, it is never the case as experimental measurements are never done fully inclusively.
Indeed they always include event selections that cut into the phase space and make the use of a full computation necessary.

Second, an amplitude can be divided into resonant and non-resonant contributions as

\begin{equation}
\mathcal{A} = \mathcal{A}_{\rm res} + \mathcal{A}_{\rm non-res} .
\end{equation}
%
But non-resonant propagators cannot be interpreted as W production.
Hence, one should only retain the resonant part of the process.
To achieve this in a gauge invariant way, one can use the double-pole approximation where only diagrams featuring a two W boson propagator are selected.
This sub-set of diagrams is then evaluation with an on-shell kinematic.
But on-shell kinematics are not unique so one should use an on-shell projection that should fulfil the following criteria:
[To be added]
In such a way, one can set a given polarisation to the W bosons in a gauge invariant manner.

Moving on to the results.
In order to be able to integrate fully over the variable $\phi$, one should consider a typical VBS event selection but without any cuts on the final state leptons.
If one consider a ${\rm W^-}$ polarised and a ${\rm W^-}$ non-polarised, one observes that the effect of the interferences between different polarisations (second term of Eq.~\eqref{eq:polarisation}) is below one per cent at the level of the total cross section.
At the level of differential distributions which do not depend on decay product of the W bosons, the same holds.
Nonetheless, looking at transverse momentum of the leptons of the angle $\phi$ of the electron, large differences arise.
Also, looking at the polarisation fraction: results of the polarisation obtained with Legendre analysis [citation?] and through direct computation agree.

Turning now to a set-up where one also applied cuts on the transverse momentum and rapidity of the final state leptons, the situation is changing drastically.
At the level of the cross section and distribution involving non W decay products, the effect is rather limited (about $2\%$).
But one can observe big differences when looking at polarisation fraction.

This analysis demonstrate that one can analysis the polarisation of massive gauge bosons in a well defined set-up for vector boson scattering.
This study has been done for the Standard Model but could be in principle extended to any new physics model.
The implementation of this method will soon be public in the code Phantom [citation?]. \\
MP: Plans for the future?

\subsection{Effective Field theory for vector-boson scattering - I. Brivio}
VBS processes represent a particularly interesting probe of new physics, as they give a unique access to the couplings of gauge bosons.
Without committing to a specific model, a convenient instrument for testing experimental data against the presence of beyond the Standard Model (BSM) effects is that of Effective Field Theories (EFTs).

In the EFT approach, the Standard Model (SM) is assumed to be the low energy limit of an unknown UV completion, whose typical scale $\Lambda$ is well separated from the electroweak one. In this scenario the new physics sector is decoupled and its impact onto observables measured at $E\sim v$ can be parameterized without specifying any property of the UV completion, by means of a Lagrangian that contains only the SM fields and respects the SM symmetries. New physics effects are organized in a Taylor expansion in $E/\Lambda$, i.e. they are encoded in an infinite series of gauge-invariant operators ordered by their canonical dimension.
This is often called SMEFT Lagrangian and, neglecting lepton number violating terms, it reads
\begin{equation}
 \mathcal{L}_{\rm SMEFT} = \mathcal{L}_{\rm SM} + \frac{1}{\Lambda^2}\mathcal{L}_{\rm dim-6} + \frac{1}{\Lambda^4}\mathcal{L}_{\rm dim-8} + \dots
\end{equation} 
with the dots standing for higher orders.
The SMEFT Lagrangian constitutes a convenient theoretical tool for probing the presence of new physics, as it provides the only systematic and model-independent parameterization of new physics effects that can be matched onto any UV completion compatible with the SM symmetries and field content.

We can restrict to leading deviations from the SM, cutting the series at dimension 6:
\begin{equation}
 \mathcal{L}_{\rm dim-6} = \sum_i C_i \mathcal{O}_i\,.
\end{equation} 
Here $\{\mathcal{O}_i\}$ is a set of gauge-invariant dimension-6 operators that form a complete basis and $\{C_i\}$ are the corresponding Wilson coefficients.
%
Any evidence for a non-zero Wilson coefficient would represent a smoking gun of new physics. Further, knowing which terms are non-vanishing can allow to characterize the new physics states and help designing more effective direct search strategies.


A complete basis for dimension-6 terms contains 59 independent structures that in complete generality are associated to 2499 independent parameters~\cite{Alonso:2013hga}. This number can be significantly reduced for example assuming CP conservation and/or an approximate flavor symmetry $U(3)^5$. Choosing convenient kinematic cuts in the experimental measurements can also help to restrict the set of relevant operators.
%
Different basis choices for $\mathcal{L}_{\rm dim-6}$ have been proposed in the literature, that are related by equation-of-motion and integration-by-parts transformations. 
Despite containing different sets of operators (often distributing the effects differently among fermions and bosons couplings!), all the bases give equivalent parameterizations for physical $S$-matrix elements, i.e. once a complete process with stable external states is computed. 
The so-called Warsaw basis~\cite{Grzadkowski:2010es} is sometimes preferred, however, due to the fact that this was the first complete basis in the literature and that its renormalisation group evolution (RGE) is completely known~\cite{Jenkins:2013zja,Jenkins:2013wua,Alonso:2013hga,Grojean:2013kd,Alonso:2014zka}.

\vskip 1em
Assuming CP conservation and a $U(3)^5$ flavor symmetry, VBS processes receive corrections from 16 dimension-6 operators. To keep the analysis as general as possible, these should be all considered simultaneously in the fit. Setting a subset of the Wilson coefficients to zero cannot be done freely as, for example, this may spoil strong correlations hidden in the parameterization and artificially remove blind directions\footnote{From a theoretical point of view, removing operators ``by hand'' is problematic because a given basis is a minimal set in which a vast amount of redundant structures have already been systematically removed. This means that each operator retained in the basis does not simply account for corrections to the couplings that it contains, but also to those contained in other structures related to it e.g. by equations of motion, that have been removed. This happens in a non-intuitive way, which is hard to control a posteriori: for instance in the Warsaw basis some operators affecting TGCs are traded for a specific combination of fermionic + Higgs terms, which are apparently unrelated to the self-couplings of the gauge bosons.}. 
A reduction of the number of parameters may be necessary, nonetheless, for the technical feasibility of the analysis. In this case the removal of some (combination of) operators may be  very carefully considered in the future.

The possibility of extending the EFT analysis with dimension-8 operators has also been discussed, as these terms can introduce important decorrelation effects between triple and quartic gauge couplings. Although this is an interesting avenue, exploring it in a consistent way is a challenging task due to the extremely large number of parameters involved and to the fact that a complete basis of dimension 8 operators is not available to date. Therefore it is advisable to defer this study to a later stage. A more compelling alternative is rather performing an analysis in the basis of the Higgs EFT (HEFT), for which complete bases have been presented in Refs.~\cite{Buchalla:2013rka,Brivio:2016fzo} (see references therein for further theoretical details and previous phenomenological studies). The HEFT differs from the SMEFT in that it is not constructed with the Higgs doublet, but rather embedding the Goldstone fields into a dimensionless matrix $\mathbf{U}=\exp(i\pi^a\sigma^a/v)$ (analogously to the pion fields in chiral perturbation theory) and treating the physical Higgs as a gauge singlet. The HEFT is more general than the SMEFT and it is matched, for instance, by theories with a strongly interacting electroweak symmetry breaking sector in the UV, such as composite Higgs models. Such an analysis would be highly motivated as the scattering of longitudinal gauge bosons constitute one of the best probes for UV scenarios matching the HEFT (see e.g.~\cite{Delgado:2013hxa,Delgado:2014jda} for recent studies), and they are among the observables that may allow to disentangle it from the SMEFT. The number of relevant Wilson coefficients for VBS is the HEFT (in the CP conserving, 4$U(3)^5$ symmetric limit) is about 30, which is larger than for the SMEFT but much lower than for including a complete dimension-8 set of operators, which makes this analysis an ideal follow-up to the SMEFT one.

One of the main points to be addressed in the EFT analysis is that of the EFT validity: as mentioned above, adopting a dimension-6 parameterization is theoretically justified only for $\Lambda$ sufficiently larger than $v$. Namely the impact of dimension-8 terms $\sim (E/\Lambda)^4$ should be roughly smaller than the experimental uncertainty. When analysing experimental data, however, the cutoff scale $\Lambda$ is unknown and the actual energy $E$ exchanged in the process is often unaccessible too. Extracting $E$ is particularly complex for VBS at the LHC, with various scales entering at different stages in the (sub-)process(es).
Thus the validity of the EFT cannot be established a priori: at best it can be verified a posteriori, basically checking that the energy range of the distributions used for the fit does not exceed the lower limit obtained for the cutoff. Some methods of this kind are already available in the literature~\cite{Contino:2016jqw,Brivio:2017ije}[ADD MORE REFS?] and could also be applied to VBS studies.
If a constraint is found to be incompatible with the validity of the EFT itself, it should be rejected.
The application of unitarisation methods should be avoided in this context: these techniques aim at correcting the divergences obtained in the kinematic distributions of montecarlo generated signals. However, these divergences are a manifestation of the breakdown of the effective expansion in the high energy regime. The damping introduced by the unitarisation effectively alters in an artificial way the behavior of the Taylor series, leading to a wrong estimation both of the EFT validity and of the constraints on the Wilson coefficients.

\vskip 1em
The first step of the EFT-VBS program, to be completed in the near future, is an accurate theoretical study of VBS in the SMEFT at dimension 6, which includes agreeing on a given parametrisation, evaluating the necessity of reducing the number of operators considered and testing the capabilities of available theoretical tools (montecarlo generators etc). This will be conducted in parallel with a preliminary study of the experimental constraints that could be obtained. One of the primary goals of these studies, in which both theorists and experimentalists will participate, is to define an optimal way to report data (cross sections and differential distributions) that maximizes the transparency and versatility of the results.
Finally, further avenues are worth exploring in a subsequent stages, among which the analysis of the HEFT basis (and later on, if possible, of dimension-8 operators) and a comparison of the impact of VBS processes with that of other datasets, with the possibility of considering a combination of different measurements in the fit.


