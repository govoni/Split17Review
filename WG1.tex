\newcommand{\MP}[1]{{ {\color{blue}{ [MP: #1]}} }}

Short introduction.
In the past years, the study of VBS from a theoretical point of view has attracted lots of interests \cite{Rauch:2016pai}.

\subsection{Complete NLO corrections to ${\rm W^+ W^+}$ scattering - M. Pellen}

The first VBS channel that has been observed during the run~I of the LHC is the same-sign WW one \cite{Aad:2014zda,Aaboud:2016ffv,Khachatryan:2014sta}.
And for run~II, a measurement has already been performed by the CMS collaboration \cite{CMS:2017adb}.
From a theoretical point of view, it is also the channel which has the smallest number of partonic channels.
Hence precise and appropriate theoretical predictions are needed to match the experimental measurements.

MP: Paragraph to be added latter.

Hence the EW corrections of order $\mathcal{O}\left(\alpha^7\right)$ are the dominant NLO contribution.
These originate from Sudakov logarithms that grow negatively large in the high energy limit.
Usually these are particularly large only in phase space regions where the  that are
Hence the impact at the level of the total fiducial cross section is rather limited.
This is not the case here where the corrections are already large at the level of the cross section and reach $-16\%$ \cite{Biedermann:2016yds}.
The origin of these large EW corrections are indeed virtual corrections and in particular the ones corresponding to the insertion of massive particles in the scattering process \cite{Biedermann:2016yds}.
As the EW corrections are particularly large, they could be measured at a high luminosity LHC, hence probe the EW sector of the Standard model.

\subsection{Monte Carlo comparisons for ${\rm W^+ W^+}$ scattering - M. Zaro}

- Citations of all the codes with name of people who ran the codes \\
Recola \cite{Actis:2012qn,Actis:2016mpe} which is based on the Collier \cite{Denner:2014gla,Denner:2016kdg} library.
MoCaNLO is a multi-channel Monte Carlo program. \\
- Description of the approximations and the table with the codes \\
- Perfect agreement at LO \\
- It seems that the VBS approximation is working well at the level of the cross section and differential distributions. \\

The results presented are only preliminary results.
In the coming months, this work will be enlarge to include comparison of predictions at NLO QCD matched to parton shower.
The QCD induced background will also studied in the same manner.
Finally, NLO EW corrections will be also included as they turn out to be large.

\subsection{Polarisation of vector bosons - E. Maina}

It has been argued that new physics model could lead to large contributions to the longitudinal polarisation of heavy gauge bosons [citation?].
Hence it is of prime importance to study the polarisation of the heavy gauge bosons and their effects both in the Standard Model and beyond.
In particular, it is very important to devise methods that allow for such studies.
Thus, a starting point is to establish these methods first for the Standard Model.
In this respect, a new method has been proposed and applied to the scattering of two W bosons of opposite charge.

First, in general, a matrix element squared can be written as

\begin{equation}
\label{eq:polarisation}
 \mathcal{A}^2 = \sum_{\lambda} \mathcal{A}^2_{\lambda, \lambda} + \sum_{\lambda \neq \lambda'} \mathcal{A}^2_{\lambda, \lambda'}, 
\end{equation}
%
where $\lambda$ represents the polarisation of the heavy gauge bosons.
Usually the last term of Eq.~\eqref{eq:polarisation} is assumed to be zero.
But this is only true when ones integrates over the full phase space in $\phi$.
In experiments, it is never the case as experimental measurements are never done fully inclusively.
Indeed they always include event selection that cut into the available phase space and make the use of a full computation necessary.

Second, an amplitude can be divided into resonant and non-resonant contributions as

\begin{equation}
\mathcal{A} = \mathcal{A}_{\rm res} + \mathcal{A}_{\rm non-res} .
\end{equation}
%
But the issue is that non-resonant propagators cannot be interpreted as W production.
Hence, one should only retain the resonant part of the process.
To achieve this in a gauge invariant one can use the double-pole approximation where only diagrams featuring a two W boson propagator will be selected.
This sub-set of diagrams is then evaluation with an on-shell kinematic.
But on-shell kinematics are not unique so one should use an on-shell projection that should fulfil the following criteria:
[To be added]
In such a way, one can applied a given polarisation to the W bosons in a gauge invariant manner.

Moving on to the results.
In order to be able to integrate fully over the variable $\phi$, one should consider a typical VBS event selection but without any cuts on the final state leptons.
If one consider a ${\rm W^-}$ polarised and a ${\rm W^-}$ non-polarised, one observes that the effect of the (second term of Eq.~\eqref{eq:polarisation}) is below one per cent at the level of the total cross section.
At the level of differential distributions which do not depend on decay product of the W bosons, the same holds.
Nonetheless, looking at transverse momentum of the leptons of the angle $phi$ of the electron, large differences arise.
Also, looking at the polarisation fraction: results of the polarisation obtained with Legendre analysis [citation?] and through direct computation agree.

Turning now to a set-up where one applied cuts on the transverse momentum and rapidity of the final state leptons, the situation is changing drastically.
At the level of the cross section and distribution involving non W decay products, the effect is rather limited (about $2\%$).
But one can observe big differences when looking at polarisation fraction.

This analysis demonstrate that one can analysis the polarisation of massive gauge bosons in a well defined set-up for vector boson scattering.
This study has been done for the Standard Model but could be in principle extended to any new physics model.
This method will soon be public in the code Phantom [citation?]. \\
MP: Plans for the future?

\subsection{Effective Field theory for vector-boson scattering - I. Brivio}

As mentioned briefly before, VBS processes could be a particularly interesting type of signature for probing new physics effects.
In this respect, EFT methods provide a proxy to test the presence of new physics in a model independent way.

One of the EFT framework is the so-called Standard Model EFT (SMEFT) which consist of adding all the dimension 6 operators to the Standard M ones.
This can be sketched as
%
\begin{equation}
 \mathcal{L}_{\rm SMEFT} = \mathcal{L}_{\rm SM} + \mathcal{L}_{\rm dim-6},
\end{equation}
%
where $\mathcal{L}$ denotes the respective Lagrangian.
In particular, one should always consider all possible operators and not only a sub-set of them.
Indeed, doing this could lead to a violation of gauge invariant which would lead to an un-physical results.
For vector boson scattering, 20 of these dimension$-6$ operators are relevant for vector-boson scattering.
This constitute a large number of operators to be fitted simultaneously but one should avoid to set some of them to zero.

An issue arising when using EFT is that it is not clear a priori to state in what range it can be applied.
In particular, VBS processes are particularly complex with various scale entering at different stage in the process.
Hence, this required a detailed analysis a posteriori of the validity of the EFT.
Some methods are already present [citation] and could probably be in VBS studies.

Also, it is not clear if EFT are of any use when one need to use unitarisation.
This seems not to be the case.

Dimension$-6$ operators are probably the first type of operators to be studied (the simplest).
Nonetheless, for VBS processes, it could be that dimension$-6$ operators could also be relevant.
The reason for this is that VBS processes involve mainly gauge bosons.



preferred basis: no from theory but popular Warsaw one.
advantages for technical reasons + complete RGE evolution.

To get constraints, you should have 10% precision -> hard for now

If a EFT coefficient is non-zero -> NP and might even tell about what is the underlying theory
first step but you want always to find the resonance!

Plan:
1. exp. constraints on EFT
- parametrisaiton for dim=6 (for a start, dim=8 is hard)
- UFO model -> test?
- SMEFT vs. HEFT
- other data set?

2. report data: flexible/model-independent
cross-section
distributions

HEFT: non-linear EFT
interesting in some cases