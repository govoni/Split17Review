\documentclass[12pt]{article}
    % General document formatting
    \usepackage[margin=0.7in]{geometry}
    \usepackage[parfill]{parskip}
    \usepackage[utf8]{inputenc}
    \usepackage{hyperref}    
    \usepackage{lineno}
    \linenumbers
    \usepackage{graphicx}
    \graphicspath{ {figures/} }
 
    \usepackage[
       backend=biber,
       style=numeric,
       sorting=none % this is to have the normal sorting!!
     ]{biblatex}
 
    \addbibresource{review.bib}

    \usepackage{authblk}
    \usepackage{tabularx}
    \usepackage{xspace}

    % Related to math
    \usepackage{amsmath,amssymb,amsfonts,amsthm}

    % macros to be used for name of particles, units, etc\dots
    \input macros

    \author[RKUH]{Christoph Falk Anders}
    \affil[RKUH]{Ruprecht-Karls-University, Heidelberg, Germany}
    \author[NBI]{Ilaria Brivio}
    \affil[NBI]{Niels Bohr International Academy and Discovery Center, Niels Bohr Institute, University of Copenhagen, Blegdamsvej 17, DK-2100, Copenhagen, Denmark}
    \author[MiB]{Pietro Govoni}
    \affil[MiB]{Milano-Bicocca University, Physics Dept.}
    \author[UHH]{Andreas Hinzmann}
    \affil[UHH]{University of Hamburg, Hamburg, Germany}
    \author[UW]{Mathieu Pellen}
    \affil[UW]{Universit\"at W\"urzburg, Institut f\"ur Theoretische Physik und Astrophysik}
    \author[LPTHE]{Marco Zaro}
    \affil[LPTHE]{Sorbonne Universit\'es, UPMC Univ. Paris 06, UMR 7589, LPTHE, F-75005, Paris, France
                      {\it and}
                   CNRS, UMR 7589, LPTHE, F-75005, Paris}
    
    \title{VBSCan Split17 workshop summary}

\begin{document}

\maketitle 

\begin{center}
\includegraphics[scale=0.1]{palace.png}
\end{center}

\abstract{
  This document summarizes the talks and discussions happened during the VBSCan Split17 workshop,
  which is the first general meeting of the VBSCan COST Action network,
  aiming at a consistent and coordinated study of vector-boson scattering
  from the phenomenological and experimental point of view, 
  for the best exploitation of the data that will be delivered by the LHC future operations.
}

\section*{Introduction}

The VBSCan COST Action Network aims at a consistent and coordinated study of vector-boson scattering (VBS)
from the phenomenological and experimental point of view, 
for the best exploitation of the data that will be delivered by the LHC future operations.
The community is organized in five working groups,
three of which are dedicated to the scientific aspects of the collaboration.
One is dedicated to the Theoretical Understanding of the VBS process,
which targets a detailed description of the VBS signal 
and relative backgrounds in the Standard Model (SM) case, 
as well as effective field theory (EFT) modeling of beyond the SM (BSM) effects.
A second one focuses on Analysis Techniques,
studying the definition of data analysis protocols and agreements 
to maximise the significance of VBS analyses at hadron colliders, 
promoting the communication between theory and experiments.
A third one fosters the optimal deployment of VBS studies 
in the hadron collider experiments data analyses.
\newline{}
The VBSCan Network is composed by theoretical and experimental physicists from both the ATLAS and CMS experiments,
as well as data analysis experts and industrial partners.
The first general meeting of the Network happened at the end of June 2017 in Split~\cite{kickoff}
and was dedicated to reviews of the data analysis status of the art,
as well as of the theoretical and experimental instruments
relevant for VBS studies.
This report contains
a summary of the talks presented\footnote{All the presentation material can be found at \url{https://indico.cern.ch/event/629638/}.} during the first general meeting of the Network,
divided into sections corresponding to the Network working groups.

\section{Theoretical Understanding}

In the past years, the study of the VBS processes attracted lots of interests in the theory community, (see \emph{e.g.}\ Ref.~\cite{Rauch:2016pai} for 
a recent review).
The activities range from providing precise predictions for the VBS signal and background processes in the SM, to studying the sensitivity of these 
processes on new physics effects, by means of effective theories or by studying the polarisation patterns of the heavy gauge bosons involved.
The program of the meeting reflects these avenues.

\subsection{Complete NLO corrections to ${\rm W^+ W^+}$ scattering}
%  M. Pellen

The first VBS process that has been observed during the run~I of the LHC is the same-sign WW production~\cite{Aad:2014zda,Aaboud:2016ffv,Khachatryan:2014sta}.
This observation has already been confirmed by a measurement of the CMS collaboration at the 13 TeV run~II~\cite{CMS:2017adb}.
In view of the growing mole of data which will be collected by the experiments, and of the consequent reduction of the uncertainties affecting these measurements, precise theoretical predictions become necessary.

In that respect next-to-leading order (NLO) QCD and electroweak (EW) corrections to such signatures should be computed.
So far, NLO computations have focused on NLO QCD corrections to the VBS process~\cite{Jager:2009xx,Jager:2011ms,Denner:2012dz,Rauch:2016pai} and its QCD-induced irreducible background process~\cite{Melia:2010bm,Melia:2011gk,Campanario:2013gea,Baglio:2014uba,Rauch:2016pai}.
No NLO EW corrections had been computed and the NLO QCD computations relied on the so-called VBS approximation.
In Ref.~\cite{Biedermann:2017bss}, for the first time, all leading order (LO) and NLO contributions to the full ${\rm p}{\rm p}\to\mu^+\nu_\mu{\rm e}^+\nu_{\rm e}{\rm j}{\rm j}$ process have been reported\footnote{The $\mathcal{O}{\left(\alpha^{7}\right)}$ corrections
were computed previously in Ref.~\cite{Biedermann:2016yds}.}.
As the full amplitudes are used, this amounts to computing three LO contributions and four NLO contributions.
At LO, the three contributions are the EW process [order $\mathcal{O}{\left(\alpha^{6}\right)}$], its QCD-induced counterpart [order $\mathcal{O}{\left(\alpha_{\rm s}^2\alpha^{4}\right)}$] as well as the interference [order $\mathcal{O}{\left(\alpha_{\rm s}\alpha^{5}\right)}$].
Due to the VBS event selection applied to the final state, the full process is dominated by the purely EW contribution (see Table~\ref{table:LOVBS}).
This EW contribution features the proper VBS diagrams but also background diagrams where, for example, the W bosons are simply radiated off the quark lines.

\begin{table}
\begin{center}
\begin{tabular}{|l||c|c|c||c|}
\hline
Order & $\mathcal{O}{\left(\alpha^{6}\right)}$ & $\mathcal{O}{\left(\alpha_{\rm s}\alpha^{5}\right)}$ & $\mathcal{O}{\left(\alpha_{\rm s}^2\alpha^{4}\right)}$ & Sum \\
\hline
\hline
${\sigma_{\mathrm{LO}}}$ [fb] 
& $1.4178(2)$
& $0.04815(2)$
& $0.17229(5)$
& $1.6383(2)$ \\
\hline
\end{tabular}
\end{center}
\caption{
Fiducial cross section from Ref.~\cite{Biedermann:2017bss} at LO for the process ${\rm p}{\rm p}\to\mu^+\nu_\mu{\rm e}^+\nu_{\rm e}{\rm j}{\rm j}$, at
orders  $\mathcal{O}{\left(\alpha^{6}\right)}$, $\mathcal{O}{\left(\alpha_{\rm s}\alpha^{5}\right)}$, and $\mathcal{O}{\left(\alpha_{\rm s}^2\alpha^{4}\right)}$.
The sum of all the LO contributions is in the last column and all contributions are expressed in femtobarn. 
The statistical uncertainty from the Monte Carlo integration on the last digit is given in parenthesis.}
\label{table:LOVBS}
\end{table}

At NLO, the four contributions arise at the orders $\mathcal{O}{\left(\alpha^{7}\right)}$, $\mathcal{O}{\left(\alpha_{\rm s}\alpha^{6}\right)}$, $\mathcal{O}{\left(\alpha_{\rm s}^{2}\alpha^{5}\right)}$, and $\mathcal{O}{\left(\alpha_{\rm s}^{3}\alpha^{4}\right)}$. 
An interesting feature is that the orders $\mathcal{O}{\left(\alpha_{\rm s}\alpha^{6}\right)}$ and $\mathcal{O}{\left(\alpha_{\rm s}^{2}\alpha^{5}\right)}$ receive both EW and QCD corrections.
Thus, at NLO (as opposed to LO) it is not possible to strictly distinguish the EW process from the QCD-induced process.
As it can be seen from Table~\ref{table:NLOVBS}, at the level of the fiducial cross section, the largest corrections are the ones of order $\mathcal{O}{\left(\alpha^{7}\right)}$.
These are the NLO EW corrections to the EW processes.

\begin{table}
\begin{center}
\begin{tabular}{|l||c|c|c|c||c|}
\hline
Order & $\mathcal{O}{\left(\alpha^{7}\right)}$ & $\mathcal{O}{\left(\alpha_{\rm s}\alpha^{6}\right)}$ & $\mathcal{O}{\left(\alpha_{\rm s}^{2}\alpha^{5}\right)}$ & $\mathcal{O}{\left(\alpha_{\rm s}^{3}\alpha^{4}\right)}$ & Sum \\
\hline
\hline 
${\delta \sigma_{\mathrm{NLO}}}$ [fb] 
& $-0.2169(3)$ 
& $-0.0568(5)$
& $-0.00032(13)$
& $-0.0063(4)$ 
& $-0.2804(7)$ \\
\hline
$\delta \sigma_{\mathrm{NLO}}/\sigma_{\rm LO}$ [\%] & $-13.2$ & $-3.5$ & $0.0$ & $-0.4$ & $-17.1$ \\
\hline
\end{tabular}
\end{center}
\caption{
NLO corrections from Ref.~\cite{Biedermann:2017bss} for the process ${\rm p}{\rm p}\to\mu^+\nu_\mu{\rm e}^+\nu_{\rm e}{\rm j}{\rm j}$ at the orders 
$\mathcal{O}{\left(\alpha^{7}\right)}$, $\mathcal{O}{\left(\alpha_{\rm s}\alpha^{6}\right)}$, $\mathcal{O}{\left(\alpha_{\rm s}^{2}\alpha^{5}\right)}$, and $\mathcal{O}{\left(\alpha_{\rm s}^{3}\alpha^{4}\right)}$.
The sum of all the NLO contributions is in the last column.
The contribution $\delta\sigma_{\mathrm{NLO}}$ corresponds to the absolute correction while $\delta \sigma_{\mathrm{NLO}}/\sigma_{\rm LO}$ gives the relative correction normalised to the sum of all LO contributions.
The absolute contributions are expressed in femtobarn while the relative ones are expressed in per cent.
The statistical uncertainty from the Monte Carlo integration on the last digit is given in parenthesis.}
\label{table:NLOVBS}
\end{table}

This is also reflected in two differential distributions in the transverse momentum for the hardest jet and invariant mass for the two leading jets in Figure~\ref{fig:VBSALL}.
At LO, the QCD-induced process as well as the interference are rather suppressed due to the typical VBS event selection (as for the fiducial cross section).
This is exemplified in the invariant mass of the two leading jets where, at high invariant mass, the QCD-induced background becomes negligible.
At NLO, the bulk of the corrections originates from the large EW corrections to the EW process at order $\mathcal{O}{\left(\alpha^{7}\right)}$.
In particular, they display the typical behaviour of Sudakov logarithms that grow large in the high-energy limit.
Note that the contribution from initial-state photon is also shown but is not taken into account in the definition of the NLO predictions.
This contribution is rather small and relatively constant in shape over the whole range studied here.
Finally, as at NLO it is not possible to isolate the EW production from its irreducible backgrounds, a global measurement of the full process ${\rm p}{\rm p}\to\mu^+\nu_\mu{\rm e}^+\nu_{\rm e}{\rm j}{\rm j}$ with all components is desirable.

\begin{figure}
\includegraphics[width=.47\textwidth]{WG1_plots/histogram_transverse_momentum_j1}
\hfill
\includegraphics[width=.47\textwidth]{WG1_plots/histogram_invariant_mass_mjj12}
% \vspace*{-1em}
\caption{Differential distributions from Ref.~\cite{Biedermann:2017bss} for ${\rm p}{\rm p}\to\mu^+\nu_\mu{\rm e}^+\nu_{\rm e}{\rm j}{\rm j}$:
transverse momentum for the hardest jet~(left) and invariant mass for the two leading jets~(right).
The two lower panels show the relative NLO corrections with respect to the full LO in per cent,
defined as $\delta_i = \delta \sigma_{i} / \sum \sigma_{\text{LO}}$, 
where $i=\mathcal{O}{\left(\alpha^{7}\right)},\mathcal{O}{\left(\alpha_{\rm s}\alpha^{6}\right)},\mathcal{O}{\left(\alpha_{\rm s}^2\alpha^{5}\right)},\mathcal{O}{\left(\alpha_{\rm s}^3\alpha^{4}\right)}$.
In addition, the NLO photon-induced contributions of order $\mathcal{O}{\left(\alpha^{7}\right)}$ is provided separately.}
\label{fig:VBSALL}
\end{figure}

We conclude by stressing that usually EW corrections are particularly large only in phase space regions which are suppressed.
Hence the impact at the level of the total fiducial cross section is usually rather limited.
This is not the case here where the corrections are already large at the level of the cross section and reach $-16\%$ \cite{Biedermann:2016yds}.
The origin of these large EW corrections are virtual corrections and in particular the ones corresponding to the insertion of massive vector particles in the scattering process \cite{Biedermann:2016yds}.
Hence, large NLO EW corrections are an intrinsic feature of VBS at the LHC.
As the EW corrections are particularly large, it might be possible to measure them at a high luminosity LHC, hence probing the EW sector of the SM to very high precision.
This is illustrated on the left-hand side of Figure~\ref{fig:VBSEW} where the band represents the estimated statistical error for a high-luminosity LHC collecting $3000~{\rm fb}^{-1}$.

\begin{figure}
\includegraphics[width=.47\textwidth]{WG1_plots/histogram_transverse_momentum_j1_ew}
\includegraphics[width=.47\textwidth]{WG1_plots/histogram_rapidity_j1j2_ew}
\caption{Differential distributions from Ref.~\cite{Biedermann:2016yds} for ${\rm p}{\rm p}\to\mu^+\nu_\mu{\rm e}^+\nu_{\rm e}{\rm j}{\rm j}$ including NLO EW corrections (upper panel) and relative NLO EW corrections (lower panel).
Left plot: Invariant-mass distribution of the four leptons.
Right plot: Rapidity distribution of the leading jet pair.
The yellow band describes the expected statistical experimental uncertainty for a high-luminosity LHC collecting $3000{\rm fb}^{-1}$ and represents a relative variation of $\pm 1/\sqrt{N_{\rm obs}}$ where $N_{\rm obs}$ is the number of observed events in each bin.}
\label{fig:VBSEW}
\end{figure}

\subsection{Monte Carlo comparisons for ${\rm W^+ W^+}$ scattering}
%  M. Zaro
In the last decade many codes capable of performing VBS simulations have appeared.
Within a network such as VBSCan it is therefore natural to perform 
a quantitative comparison of these programs, both to cross-validate results and to assess the impact of the different approximations which are used.
In fact, already at LO, when considering the process ${\rm p}{\rm p}\to\mu^+\nu_\mu{\rm e}^+\nu_{\rm e}{\rm j}{\rm j}$ at order $\mathcal O (\alpha^6)$, the various implementations of VBS simulations are different.
They differ, for example,
by the (non-)inclusion of diagrams with vector bosons in the $s$-channel or by the treatment of interferences between diagrams.
The reason of these differences is that, when typical signal cuts for VBS are imposed, these effects turn out to be small on rates and distributions.

In the comparison, the following codes are used: 

\begin{itemize}
 \item The program {\sc Bonsay} consists of a general-purpose Monte Carlo integrator
and matrix elements taken from several sources: Born matrix elements are
adapted from the program {\sc Lusifer} \cite{Dittmaier:2002ap} for the partonic
processes, real matrix elements are written by Marina Billoni, and virtual
matrix elements by Stefan Dittmaier.
One loop integrals are evaluated using the {\sc Collier} library
\cite{Denner:2014gla,Denner:2016kdg}.

  \item {\sc MadGraph5\_aMC@NLO}~\cite{Alwall:2014hca} is an automatic meta-code, automatically generating the source code to simulate any scattering process
      including NLO QCD corrections both at fixed order and including matching to parton showers. It makes use of the FKS subtraction method~\cite{Frixione:1995ms,
        Frixione:1997np} (automated in the module {\sc MadFKS}~\cite{Frederix:2009yq,
        Frederix:2016rdc}) for regulating IR singularities. The computations of one-loop amplitudes are carried out by switching dynamically between 
        two integral-reduction techniques, OPP~\cite{Ossola:2006us} or Laurent-series expansion~\cite{Mastrolia:2012bu},
        and TIR~\cite{Passarino:1978jh,Davydychev:1991va,Denner:2005nn}. These have been automated in the module {\sc MadLoop}~\cite{Hirschi:2011pa}, which 
        in turn exploits {\sc CutTools}~\cite{Ossola:2007ax}, {\sc Ninja}~\cite{Peraro:2014cba,
        Hirschi:2016mdz}, or {\sc IREGI}~\cite{ShaoIREGI}, together with an in-house implementation of the {\sc OpenLoops} optimisation~\cite{Cascioli:2011va}.\\
        The simulation of VBS at NLO-QCD accuracy can be performed by issuing the following commands in the program interface:
\begin{verbatim}
> set complex_mass_scheme #1
> import model loop_qcd_qed_sm_Gmu #2
> generate p p > e+ ve mu+ vm j j QCD=0 [QCD] #3
> output #4
\end{verbatim}
  With these commands the complex-mass scheme is turned on {\tt \#1}, then the NLO-capable model is loaded {\tt \#2}\footnote{Despite
            the {\tt loop\_qcd\_qed\_sm\_Gmu} model also includes NLO counterterms for computing electro-weak corrections, it is not yet possible to compute such corrections 
        with the current version of the code.}, finally the process code is generated {\tt \#3} (note the {\tt QCD=0} syntax to select the purely-electroweak process)
        and written to disk {\tt \#4}. Because of some internal limitations, which will be lifted in the future version capable of computing both QCD and EW corrections, 
        only loops with QCD-interacting particles are generated.
  \item The {\sc Powheg-Box}~\cite{Alioli:2010xd,Frixione:2007vw} is a framework for matching NLO-QCD calculations with parton showers.
It relies on the user providing the matrix elements and Born phase space, but will automaticaly construct FKS %\cite{Frixione:1995ms} 
subtraction terms and the phase space for the real emission.
For the VBS processes all matrix elements are being provided by a previous version of {\sc VBFNLO}~\cite{Arnold:2008rz, Arnold:2011wj, Baglio:2014uba} and hence the approximations used in the {\sc Powheg-Box} are the similar to those used in {\sc VBFNLO}.

  \item The program {\sc Recola+MoCaNLO} is made of a flexible Monte Carlo program dubbed {\sc MoCaNLO}~\cite{MoCaNLO} and the general matrix element generator {\sc Recola} \cite{Actis:2012qn,Actis:2016mpe}.
To numerically evaluate the one-loop scalar and tensor integrals, {\sc Recola} relies on the {\sc Collier} library \cite{Denner:2014gla,Denner:2016kdg},
These tools have been successfully used for the computation of the full NLO corrections for VBS~\cite{Biedermann:2016yds,Biedermann:2017bss}.

  \item {\sc VBFNLO}~\cite{Arnold:2008rz, Arnold:2011wj, Baglio:2014uba} is a flexible
parton-level Monte Carlo for processes with electroweak bosons. It
allows the calculation of VBS processes at NLO QCD in the VBF
approximation also including the s-channel triboson contribution,
neglecting interferences between the two. Besides the SM, also anomalous
couplings of the Higgs and gauge bosons can be simulated.

  \item {\sc Whizard}~\cite{Moretti:2001zz,Kilian:2007gr} is a multi-purpose
event generator with the LO matrix element generator {\sc O'Mega}. It
provides FKS subtraction terms for any NLO process, while virtual matrix
elements are provided externally by {\sc
OpenLoops}~\cite{Cascioli:2011va} (alternatively, {\sc Recola}~\cite{Actis:2012qn,Actis:2016mpe}
can be used as well). {\sc Whizard} allows to simulate a
huge number of BSM models as well, in particular for new physics in
the VBS channel in terms of both higher-dimensional operators as well as explicit
resonances.

\end{itemize}

The complete comparison of the codes will be published in a separate work. Here, we present some preliminary results obtained at LO $\mathcal O (\alpha^6)$ and including
NLO QCD corrections at fixed-order $\mathcal O (\alpha^6\alpha_s)$, for the process ${\rm p}{\rm p}\to\mu^+\nu_\mu{\rm e}^+\nu_{\rm e}{\rm j}{\rm j}$.
In Table~\ref{tab:wg1_codes} the details of the various codes are reported. In particular, it is specified whether:
\begin{itemize}
    \item all $s-$ and $t/u-$channel diagrams that lead to the considered final state are included;
    \item interferences between diagrams are included at LO;
    \item diagrams which do not feature two resonant vector bosons are included;
    \item the so-called non-factorisable (NF) QCD corrections, that is the corrections where (real or virtual) gluons are exchanged between different quark lines,
        are included;
    \item EW corrections to the $\mathcal O (\alpha^5\alpha_s)$ interference are included. These corrections are of the same order as the NLO QCD corrections to
        the  $\mathcal O (\alpha^6$) term.
\end{itemize}
%
\begin{table}
    \footnotesize
    \begin{tabularx}{\textwidth}{c|c|X|X|X|X|X}
        Contact person  &  Code  &  $\mathcal O(\alpha^6)$ $|s|^2/$ $|t|^2/|u|^2$  &  $\mathcal O(\alpha^6)$ interf.  &  Non-res.  &  NF QCD  &  EW corr. to $\mathcal O(\alpha^5\alpha_s)$  \\
        \hline
        \hline
        A. Karlberg  &  {\sc POWHEG}  &  $t/u$  &  No  &  Yes  &  No  &  No  \\
        M. Pellen    &  {\sc Recola+MoCaNLO}  &  Yes  &  Yes  &  Yes  &  Yes  &  Yes  \\
        M. Rauch     &  {\sc VBFNLO}  &  Yes  &  No  &  Yes  &  No  &  No  \\
        C. Schwan    &  {\sc Bonsay}  &  $t/u$  &  No  &  Yes, virt. No  &  No  &  No  \\
        M. Zaro      &  {\sc MG5\_aMC}  &  Yes  &  Yes  &  No virt.  &  No  &  No \\
        V. Rothe     &  {\sc Whizard}  &  Yes  &  Yes  &  Yes  &  Yes  &  Yes \\  
    \end{tabularx}
    \caption{\label{tab:wg1_codes} Summary of the different properties of the codes employed in the comparison.}
\end{table}
%

We simulate VBS production at the LHC, with a center-of-mass energy $\sqrt s = 13 \TeV$. We assume five massless flavours in the proton, and employ the NNPDF~3.0 parton 
density~\cite{Ball:2014uwa}
with NLO QCD evolution (the {\tt lhaid} in LHAPDF6~\cite{Buckley:2014ana} for this set is 260000) and strong coupling constant $\alpha_s\left( \MZ \right) = 0.118$. Since
the employed PDF set has no photonic density, photon-induced processes are not considered. Initial-state collinear singularities are factorised with the  ${\overline{\rm MS}}$ 
scheme, consistently with what is done in NNPDF.\\
We use the following values for the mass and width of the massive particles:
% 
\begin{alignat}{2}
                  \Mt   &=  173.21\GeV,       & \quad \quad \quad \Gt &= 0 \GeV,  \nonumber \\
                \MZOS &=  91.1876\GeV,      & \quad \quad \quad \GZOS &= 2.4952\GeV,  \nonumber \\
                \MWOS &=  80.385\GeV,       & \GWOS &= 2.085\GeV,  \nonumber \\
                M_{\rm H} &=  125.0\GeV,       &  \GH   &=  4.07 \times 10^{-3}\GeV.
\end{alignat}
%
The pole masses and widths of the W and Z~bosons are obtained from the measured on-shell (OS) values \cite{Bardin:1988xt} according to
%
\begin{equation}
M_V = \frac{\MVOS}{\sqrt{1+(\GVOS/\MVOS)^2}},\qquad  
\Gamma_V = \frac{\GVOS}{\sqrt{1+(\GVOS/\MVOS)^2}}.
\end{equation}
%
The EW coupling is renormalised in the $G_\mu$ scheme \cite{Denner:2000bj} where
\begin{equation}
    G_{\mu}    = 1.16637\times 10^{-5}\GeV^{-2}.
\end{equation}
The derived value of the EW coupling $\alpha$, corresponding to our choice of input parameters, is 
\begin{equation}
 \alpha = 7.555310522369 \times 10^{-3}. \\
\end{equation}
We employ the complex-mass scheme~\cite{Denner:1999gp,Denner:2005fg} to treat unstable intermediate particles in a gauge-invariant manner.\\

Cross sections and distribution are computed within the following VBS cuts inspired from experimental measurements \cite{Aad:2014zda,Aaboud:2016ffv,Khachatryan:2014sta,CMS:2017adb}: 
\begin{itemize}
    \item The two same-sign charged leptons are required to have
        \begin{align}
         \ptsub{\Pl} >  20\GeV,\qquad |y_{\Pl}| < 2.5, \qquad \Delta R_{\Pl\Pl}> 0.3\,.
        \end{align}
    \item The total missing transverse energy, computed from the vectorial sum of the transverse momenta of the two neutrinos in the event,
        is required to be
        \begin{align}
          \etsub{\text{miss}}=p_{\rm T, miss} >  40\GeV\,.
        \end{align}
    \item QCD partons (quark and gluons) are clustered together using the anti-$k_T$ algorithm~\cite{Cacciari:2008gp} with distance parameter $R=0.4$. Jets are required
        to have
        \begin{align}
         \ptsub{\Pj} >  30\GeV, \qquad |y_\Pj| < 4.5, \qquad \Delta R_{\Pj\Pl} > 0.3 \,.
        \end{align}
        On the two jets with largest transverse-momentum the following invariant-mass and rapidity-separation cuts are imposed
        \begin{align}
         m_{\Pj \Pj} >  500\GeV,\qquad |\Delta y_{\Pj \Pj}| > 2.5.
        \end{align}
%         Finally, all jest in the event are required to be separated from charged leptons:
%         \begin{align}
%          \qquad\Delta R_{\Pj\Pl} > 0.3 .
%         \end{align}
    \item When EW corrections are computed, real photons and charged fermion are clustered together using the anti-$k_T$ algorithm with 
        radius parameter $R=0.1$. In this case, leptons and quarks mentioned above must be understood as {\it dressed fermions}. Photons
        which are not combined at this step are clustered with QCD partons to form jets as it is described previously.
\end{itemize}

\begin{table}[h!]
    \centering
    \begin{tabular}{c|c}
        Code  &  $\sigma[\rm{fb}]$  \\
        \hline
        \hline
        {\sc Bonsay}  &  $1.5524 \pm 0.0002$ \\
        {\sc MG5\_aMC}&  $1.547 \pm 0.001$  \\ 
        {\sc POWHEG}  &  $1.5573 \pm 0.0003$ \\
        {\sc Recola+MoCaNLO}  &  $1.5503 \pm 0.0003$ \\
        {\sc VBFNLO}  &  $1.5538 \pm 0.0002$ \\
        {\sc Whizard}&  $ 1.5539 \pm 0.0004 $   
    \end{tabular}
    \caption{\label{tab:wg1_LOrates} Rates at LO accuracy within VBS cuts obtained with the different codes used in this comparison, 
    for the ${\rm p}{\rm p}\to\mu^+\nu_\mu{\rm e}^+\nu_{\rm e}{\rm j}{\rm j}$ process.}
\end{table}
%
\begin{figure}[h!]
   \centering
   \includegraphics[width=0.49\textwidth,angle=0,clip=true,trim={0.4cm 2.5cm 0.cm 1.cm}]{figures/mjj_LO.pdf}
   \includegraphics[width=0.49\textwidth,angle=0,clip=true,trim={0.4cm 2.5cm 0.cm 1.cm}]{figures/mll_LO.pdf}
\caption{\label{fig:wg1_mjj-llLO} Invariant-mass of the two tagging jets (left) and of the two leptons (right), at LO accuracy, 
computed with the different codes used in this comparison. The inset shows the ratio over {\sc VBFNLO}.
}
\end{figure}
%
\begin{table}[h!]
    \centering
    \begin{tabular}{c|c|c|c}
        Code  &  $\sigma[\rm{fb}]$  \\
        \hline
        \hline
        {\sc Bonsay}  &  $1.3366 \pm 0.0009$  \\
        {\sc MG5\_aMC}&  $1.318  \pm 0.003$  \\
        {\sc POWHEG}  &  $1.334 \pm 0.0003$  \\
        {\sc Recola+MoCaNLO}  &  $1.317 \pm 0.004 $ \\
        {\sc VBFNLO}  &  $1.3531 \pm 0.0003$  \\
    \end{tabular}
    \caption{\label{tab:wg1_NLOrates} Rates at NLO-QCD accuracy within VBS cuts obtained with the different codes used in this comparison, 
    for the ${\rm p}{\rm p}\to\mu^+\nu_\mu{\rm e}^+\nu_{\rm e}{\rm j}{\rm j}$ process.}
\end{table}
In Table~\ref{tab:wg1_LOrates} we report the total rates at LO accuracy obtained with the set-up described above, and in Figure~\ref{fig:wg1_mjj-llLO} we show the results
for the tagging-jet (left) and lepton-pair (right) invariant-mass distribution. In both case we show the absolute distributions in the main frame of the 
figures, while in the inset the ratio over {\sc VBFNLO} is displayed. For both observables we find 
a relatively good agreement among the various tools, which confirms the fact
that contributions from $s$-channel diagrams as well as from non-resonant configurations are strongly suppressed in the fiducial region.
At NLO, rates show slightly larger discrepancies, as it can be observed in Table~\ref{tab:wg1_NLOrates}. This is most likely due to low dijet invariant-mass configurations, where
$s$-channel diagrams and interferences are less suppressed than at LO, because of the presence of extra QCD radiation.

We conclude this section by recalling that the results presented must be regarded as preliminary.
In the coming months, this work will be enlarged to include comparison of predictions at NLO QCD matched to parton shower or with EW corrections, 
as well as to study the effect of changing 
the VBS cuts. The QCD-induced background will also be studied.

\subsection{Polarisation of vector bosons}
% E. Maina

Processes related to new physics could disturb the delicate balance which preserves unitarity
in VBS  between longitudinally polarised vector bosons
and lead to potentially large enhancements of the VBS rate, making it the ideal process for searches of 
deviations from the SM and hints of New Physics. 
Developing methods which allow the separation of the different vector polarisations is, therefore, of primary
relevance.
A new technique has been proposed and applied to the scattering of two W bosons of opposite charge
\cite{Ballestrero:2017bxn}.

The basic tool has been known for a long time.
The polarisation tensor in the W propagator can be expressed in terms of polarisation vectors:
\begin{equation}
-g^{\mu\nu} + \frac{k^{\mu}k^{\nu}}{M^2} = \sum_{\lambda = 1}^4 \varepsilon^{\mu}_\lambda(k) 
\varepsilon^{\nu^*}_{\lambda}(k)\,\,.
\end{equation}

The decay amplitudes of the W depend on its polarisation.
In the rest frame of the $\ell\nu$ pair, they are:
\begin{equation}\label{eq:longamp}
\mathcal{M^D}_0 = ig\,\sqrt{2}E \,\sin\theta \,\,,\quad
\mathcal{M^D}_{R/L} = ig\,E \,(1 \pm \cos\theta)e^{\pm i\phi} \,\,,
\end{equation}
 where $0,R,L$ refer to the longitudinal, right and left polarisations and $(\theta, \phi)$ are the charged 
 lepton angles relative to the boson direction.
Hence, each physical polarisation is uniquely associated with a specific angular distribution of the charged
lepton.    

Two issues, however, remain unresolved: 

\begin{itemize}
\item With few exceptions, electroweak boson production processses are described by amplitudes
including non resonant diagrams, which cannot be interpreted as
production times decay of any vector boson.
These diagrams are essential for gauge invariance and cannot be ignored.
For them, separating polarisations is simply unfeasible. 
\item Since the  W's are unstable particles, the decays of the indidual 
polarisations interfere among themselves.
 \end{itemize} 

In order to define amplitudes with definite $W$ polarisation, it is necessary to devise an accurate 
approximation to the full result that only involves double resonant diagrams.
Reference~\cite{Ballestrero:2017bxn} employed an on-shell projection (OSP) method,
similar to the procedure employed for
the calculation of EW radiative corrections to W$^+$W$^-$ production in
reference~\cite{Denner:2000bj}.
 
It consists in substituting the momentum of the $\ell\nu$ pair  with a momentum on the W mass shell, 
while the denominator in the W propagator is left untouched. However,
this projection is not uniquely defined. In order to have an unambiguous
prescription one can choose to conserve:
the total four--momentum of the WW system (thus, also $M_{WW}$ is conserved);
the direction of the two W bosons in the WW center of mass frame;
the angles of each charged lepton, in the corresponding W center of mass frame, relative to the boson
direction in the lab.
This procedure is gauge invariant. 

 If, for instance, one considers a polarised W$^-$
 boson and a non-polarised W$^+$ one, once all non double resonant diagrams have been dropped and the
 resonant ones have been projected,
the squared amplitude becomes:
\begin{equation}\label{eq:interfpol}
\underbrace{\left|\mathcal{M}\right|^2}_{\textrm{coherent sum}} = \underbrace{\sum_{\lambda}\left|
\mathcal{M}_{\lambda}\right|^2}_{\textrm{incoherent sum}} + \underbrace{\sum_{\lambda \neq \lambda'}
\mathcal{M}_{\lambda}^{ *}\mathcal{M}_{\lambda'}}_{\textrm{interference terms}}\,.
\end{equation}
where $\lambda$ is the W$^-$ polarisation.
In the absence of cuts on the final state leptons, the interference terms in Equation~\ref{eq:interfpol} cancel upon 
integration and the projected cross section is simply the sum of singly polarised cross sections. 
In the W center of mass frame the charged lepton angular distribution is: 
\begin{equation}
\frac{1}{\sigma(X)} \,\,\frac{d\sigma(\theta,X)}{d\cos\theta}
\ =\  \frac{3}{8} (1 + \cos\theta)^2 \,f_L
    + \frac{3}{8} (1 - \cos\theta)^2 \,f_R
    + \frac{3}{4} \sin^2\theta \, f_0 \,,
\label{eq:dcdist}
\end{equation}

The coherent sum of  OSP polarised amplitudes differs by about one percent from the exact cross section.
The differential distributions which do not depend on decay products of the
W bosons are equally well described.
Other variables, like the transverse momentum or the
angle $\phi$ of the electron, show sizable differences  between the exact distributions and the projected 
ones.
The polarisation fractions obtained with a Legendre analysis and through direct computation of singly 
polarised processes agree.

\begin{figure}[!tb]
\centering
\subfigure[{$\cos\theta_e$}\label{fig:distrib_thetal_lepcut}]
{\includegraphics[scale=0.34]{WG1_plots/thetael_mww300_polOSPvsFUL_lepcut20_withLegendre.pdf}}
\qquad
\subfigure[{polarisation fractions}\label{fig:polfrac_MWW_lepcut}]
{\includegraphics[scale=0.34]{WG1_plots/polfracVSmww_H125_PA_lepcut20.pdf}}
\caption{Distribution of $\cos\theta$ in the W reference frame (left), polarisation fractions as functions of 
$M_{WW}$ (right).}
\label{fig:distrib_thetal_polfrac_lepcut}
\end{figure} 

The approach proposed in ref~\cite{Ballestrero:2017bxn} can be applied also when
standard acceptance cuts ($p_t^{\ell}>20 \,\,\mathrm{GeV}$ and
 $|\eta^{\ell}| < 2.5$) are imposed on both charged leptons.

The coherent sum of  OSP polarised amplitudes differs from the full result by about 2\%.
The interference among polarisations is generally small. 
 
The distributions obtained from the incoherent sum of three OSP distributions agree  well with the full 
result for variables which do not depend on the W decay products.
The individual polarisations are not affected equally by the cuts. Typically, the cross section for right--
handed W's 
is reduced the most, followed by the cross section for longitudinally polarised W's. Left--handed W 
bosons seem to be the least sensitive to acceptance cuts.

The full angular distribution, Figure~\ref{fig:distrib_thetal_lepcut},
is approximated within a few percent, over the full range, by the sum of the 
unpolarised distributions. 
The exact result, shown by the black histogram, however, is not of the form of equation~\ref{eq:dcdist}.
This becomes clear expanding the full result on the first three Legendre polynomials,
which yields the blue, green, and orange smooth 
curves in Figure~\ref{fig:distrib_thetal_polfrac_lepcut}. Their sum is the smooth black curve which fails to describe 
the correct distribution.
Even in the presence of cuts, the fraction of longitudinally polarised W$^-$'s is well above 10\%.

The fact that the exact distribution is well described by the incoherent sum of the polarised differential 
distributions
allows for a measurement of the polarisation fractions, within a single model,
even in the presence of cuts on the charged
leptons, using Montecarlo templates for the fit. 

This analysis demonstrates that it is possible to study the polarisation of massive gauge
bosons in a well defined set-up for vector-boson scattering processes. The  method has been 
implemented in the code Phantom \cite{Ballestrero:2007xq}.


\subsection{Effective Field theory for vector-boson scattering}
% I. Brivio

VBS processes represent a particularly interesting probe of new physics, as they give a unique access to the couplings of gauge bosons.
Without committing to a specific model, a convenient instrument for testing experimental data against the presence of BSM effects is that of Effective Field Theories (EFTs).

In a EFT approach, the SM is assumed to be the low energy limit of an unknown UV completion, whose typical scale $\Lambda$ is well separated from the electroweak one.
In this scenario the new physics sector is decoupled and its impact onto observables measured at $E\sim v$ can be parametrised without specifying any property of the UV completion, by means of a Lagrangian that contains only the SM fields and respects the SM symmetries.
New physics effects are organized in a Taylor expansion in $E/\Lambda$, \emph{i.e.}\ they are encoded in an infinite series of gauge-invariant operators ordered by their canonical dimension.
This is often called SMEFT (SM EFT) Lagrangian and, neglecting lepton number violating terms, it reads
\begin{equation}
 \mathcal{L}_{\rm SMEFT} = \mathcal{L}_{\rm SM} + \frac{1}{\Lambda^2}\mathcal{L}_{\rm dim-6} + \frac{1}{\Lambda^4}\mathcal{L}_{\rm dim-8} + \dots
\end{equation} 
with the dots standing for higher orders.
The SMEFT Lagrangian constitutes a convenient theoretical tool for probing the presence of new physics, as it provides the only systematic and model-independent parametrisation of new physics effects that can be matched onto any UV completion compatible with the SM symmetries and field content.

We can restrict to leading deviations from the SM cutting the series at dimension 6 which reads
\begin{equation}
 \mathcal{L}_{\rm dim-6} = \sum_i C_i \mathcal{O}_i\, .
\end{equation} 
Here $\{\mathcal{O}_i\}$ is a set of gauge-invariant dimension-6 operators that form a complete basis and $\{C_i\}$ are the corresponding Wilson coefficients.
%
Any evidence for a non-zero Wilson coefficient would represent a smoking gun of new physics.
Further, knowing which terms are non-vanishing can allow to characterise the new physics states and help designing more effective direct search strategies.


A complete basis for dimension-6 terms contains 59 independent structures (+ their hermitian conjugates) that in complete generality are associated to 2499 independent parameters~\cite{Alonso:2013hga}.
This number can be significantly reduced by assuming CP conservation and/or an approximate $U(3)^5$ flavour symmetry.
Choosing convenient kinematic cuts in the experimental measurements can also help to restrict the set of relevant operators.
Different basis choices for $\mathcal{L}_{\rm dim-6}$ have been proposed in the literature, that are related by equation-of-motion and integration-by-parts transformations. 
Despite containing different sets of operators (often distributing the effects differently among fermions and bosons couplings), all the bases give equivalent parametrisations for physical $S$-matrix elements, \emph{i.e.}\ once a complete process with stable external states is computed. 
The so-called Warsaw basis~\cite{Grzadkowski:2010es} is sometimes preferred, due to the fact that this was the first complete basis in the literature and that its renormalisation group evolution (RGE) and one-loop finite renormalisation are completely known~\cite{Jenkins:2013zja,Jenkins:2013wua,Alonso:2013hga,Grojean:2013kd,Alonso:2014zka,Ghezzi:2015vva}.


% \vskip 1em
Assuming CP conservation and a $U(3)^5$ flavour symmetry, VBS processes receive corrections from 16 dimension-6 operators.
To keep the analysis as general as possible and to have a well-defined IR limit of a given underlying UV sector, these should be all considered simultaneously in the fit.
Setting a subset of the Wilson coefficients to zero cannot be done arbitrarily.
For example, this may spoil strong correlations hidden in the parametrisation and artificially remove blind 
directions\footnote{From a theoretical point of view, removing operators arbitrarily is problematic because a given basis is a minimal set in which a vast amount of redundant structures have already been systematically removed.
This means that each operator retained in the basis does not simply account for corrections to the couplings that it contains, but also to those contained in other structures related to it \emph{e.g.}\ by equations of motion, that have been removed.
This happens in a non-intuitive way, which is hard to control a posteriori.
For instance in the Warsaw basis some operators affecting triple gauge couplings (TGCs) are traded for a specific combination of fermionic + Higgs terms, which are apparently unrelated to the self-couplings of the gauge bosons.}.
In particular, including anomalous fermion couplings may have a significant impact on the analysis, despite the strong constraints imposed by LEP measurements (see \emph{e.g.}\ reference~\cite{Baglio:2017bfe} for a recent study in the context of W$^+$ W$^-$ production at the LHC).
A reduction of the number of parameters may be necessary, nonetheless, for the technical feasibility of the analysis. In this case the removal of some (combination of) operators may be  very carefully considered in the future.

The possibility of extending the EFT analysis with dimension-8 operators has also been discussed, as these terms can introduce important decorrelation effects between triple and quartic gauge couplings.
Although this is an interesting avenue, exploring it in a consistent way is a challenging task due to the extremely large number of parameters involved 
(considering one fermion generation, there are 895 B-conserving independent operators at $d=8$, among which up to 86 can contribute to quartic gauge couplings (QGCs) and TGCs~\cite{Henning:2015alf}) 
and to the fact that a complete basis of dimension-8 operators is not available to date. 
Therefore it is advisable to defer this study to a later stage. A more compelling alternative is rather performing an analysis in the basis of the Higgs EFT (HEFT), 
for which complete bases have been presented in Refs.~\cite{Buchalla:2013rka,Brivio:2016fzo} (see references therein for further theoretical details and previous phenomenological studies).
The HEFT differs from the SMEFT in that it is not 
constructed with the Higgs doublet, but rather embedding the Goldstone fields into a dimensionless matrix $\mathbf{U}=\exp(i\pi^a\sigma^a/v)$ (analogously to the pion fields in chiral perturbation theory) and treating the physical Higgs as a gauge singlet. The HEFT is more general than the SMEFT and it matches the low energy limit, for instance, of some theories with a strongly interacting electroweak symmetry breaking sector in the UV, such as composite Higgs models.
Such an analysis would be highly motivated as the scattering of longitudinal gauge bosons constitutes one of the best probes for UV scenarios matching the HEFT (see \emph{e.g.}\ Refs.~\cite{Delgado:2013hxa,Delgado:2014jda} for recent studies), and they are among the observables that may allow to disentangle it from the SMEFT. The number of relevant Wilson coefficients for VBS in the HEFT (in the CP conserving, $U(3)^5$ symmetric limit) is about 30, which is larger than for the SMEFT but much lower than for including a complete dimension-8 set of operators, 
which makes this analysis an ideal follow-up to the SMEFT one.

One of the main points to be addressed in the EFT analysis is that of the EFT validity: as mentioned above, adopting a dimension-6 parametrisation is theoretically justified only for $\Lambda$ sufficiently larger than $v$. Namely the impact of dimension-8 terms $\sim (E/\Lambda)^4$ should be roughly smaller than the experimental uncertainty. When analysing experimental data, however, the cutoff scale $\Lambda$ is unknown and the actual energy $E$ exchanged in the process is often unaccessible too. Extracting $E$ is particularly complex for VBS at the LHC, with various scales entering at different stages in the (sub-)process(es).
Thus the validity of the EFT cannot be established a priori: at best it can be verified a posteriori, checking that the energy range of the distributions used for the fit does not exceed the lower limit obtained for the cutoff. Some methods of this kind have been discussed in the literature (see e.g.~\cite{Busoni:2013lha,Buchmueller:2013dya,Biekoetter:2014jwa,Englert:2014cva,Racco:2015dxa,Contino:2016jqw,Brivio:2017ije}) and could also be applied to VBS studies.
If a constraint is found to be incompatible with the validity of the EFT itself, it should be rejected.
Attention should be paid to the application of unitarisation methods, that are often employed to correct the divergences obtained in the kinematic distributions of Monte Carlo generated signals. Introducing a damping of the distribution tails, these techniques may alter the behavior of the Taylor series in a way that does not reflect the correct behavior of the EFT at high energies (which is indeed divergent where the expansion breaks down) and lead to an incorrect estimation of the constraints.

The first step of the EFT-VBS program is an accurate theoretical study of VBS in the SMEFT at dimension-6, which includes agreeing on a given parametrisation, evaluating the necessity of reducing the number of operators considered and testing the capabilities of available theoretical tools (Monte Carlo generators etc).
This will be conducted in parallel with a preliminary study of the experimental constraints that could be obtained. One of the primary goals of these studies, in which both theorists and experimentalists will participate, is to define an optimal way to report data (cross sections and differential distributions) that maximizes the transparency and versatility of the results.
Finally, further avenues are worth exploring in a subsequent stages, among which the analysis of the HEFT basis (and later on, if possible, of dimension-8 operators) and a comparison of the impact of VBS processes with that of other datasets, with the possibility of considering a combination of different measurements in the fit.





\section{Analysis Techniques}

\label{WG2}

\subsection{Experimental Overview -- N. Lorenzo Martinez}
\begin{itemize}
\item Blabla overview of experimental resuls
\item Highlight 5 sigma in ww
\item More channels getting covered => important for EFT interpretation: different channels sensitive to different couplings.
\item Would be good to have more complete coverage from both experiments
\item Notable difference between analysis when treating unitarization => hurdle for combinations of EFT interpretation
\item Notable differences between analysis when treating EWK QCD interference => hurdle for comparison/combination of SM results.
\end{itemize}

table: Add references?!
\begin{table}
  \begin{tabular}{|l|c|c|}
    \hline
    channel & ATLAS & CMS \\
    \hline
    $Z(\ell\ell)\gamma$ & \checkmark & \checkmark \\
    $Z(\nu\nu)\gamma$ & \checkmark & $\times$ \\
    $W^\pm W^\pm$ &\checkmark & \checkmark \\
    $W(\ell\nu)\gamma$ & $\times$ &\checkmark \\
    $Z(\ell\ell)Z(\ell\ell)$&  $\times$  &\checkmark \\
    $W(\ell\nu)Z(\ell\ell)$ & \checkmark & $\times$ \\
    $W(\ell\nu)V(qq)$ & \checkmark & $\times$ \\
    \hline
  \end{tabular}  
\end{table}


\subsection{Common Selection Criteria -- X. Janssen}

Goal have realistic phase-space definition for theory- and feasability studies

Comparison of selections of existing analysis.
Reasonable baseline possible for ``easy'' channels (e.g. WW), where signal is accessible in cut \& count style analysis.

Ultimately difficult for signals with very low cross-section / BR / efficiency (e.g. ZZ), as these require very loose phase-space selection + sophisticated MVA final selection, which is not easily modeled for the above purpose.

\subsection{Prospects -- M. Kobel}

Major topics:
Longitudinal component.
purest channel (WW), polarization is not accessible. Channels with easy pol. typically have high QCD contributions.
Some ideas to improve by devising polarization depndent variables.

BSM interpretations.
Advantages / Disadvantages of EFTs vs more explicit resonance modeling.
EFT advantage: generic, complete, computations avaliable.
EFT disadvantage: unitarization issues. Poor modeling of resonance tails (i.e. the ``$<<$'' in the usual assumption of $<<\Lambda$ in EFTs can be ver large)

Need some figures here?


\section{Experimental Measurements}

\label{WG3}

\subsection{Large R jets and boosted object tagging in ATLAS}
% C. F. Anders

At the high center of mass energies of the Large Hadron Collider even the heaviest known SM particles can be observed with large transverse momenta, in the so called boosted topology. Boosted $W$ and Higgs bosons and top quarks that decay to quarks will be highly collimated and can therefore be reconstructed in a single jet with large radius parameter $R$. In ATLAS the jets are reconstructed with the anti-k$_t$~\cite{Cacciari:2008gp} algorithm, usually requiring a transverse momentum of $p_T>200$~GeV and a radius parameter of $R=1.0$ and are trimmed~\cite{Krohn:2009th} with the parameters  $R_{\mathrm{sub}}=$ 0.2 and $f_{\mathrm{cut}}=$ 5\%. 

To distinguish signal, e.g.\ real $W$ bosons from QCD induced jet backgrounds one of the main observables is the jet mass, calculated from the jet constituents. In Figure \ref{fig:jetmass_W} the distribution of the jet mass in data and simulation is shown after a lepton + jet selection that aims at selecting $t\bar{t}$ events. It shows a clear peak at the expected SM $W$ boson mass. Using a further discriminating variable $W$ boson taggers are build that have a fixed signal efficiency of 50\% and reduce the background by a factor of 50.
\begin{figure}[!h]
\begin{centering}
\includegraphics[width=0.48\textwidth]{WG3_plots/fig_07}
\caption{Distribution of the calorimeter jet mass spectrum for the leading-p$_T$ jet in 13 TeV data and MC simulation using trimmed~\cite{Krohn:2009th} anti-k$_t$ R=1.0 jets with trimming parameters $f_{\mathrm{cut}}=5$\% and $R_{\mathrm{sub}} = 0.2$ in lepton+jets events. The large $R$ jets are required to have $p_T>200$~GeV~\cite{ATLASplots1}.}
\label{fig:jetmass_W}
\end{centering}
\end{figure}

Recently, the possibilities of adding more information and exploiting multi-dimensional corelations has been explored by using Boosted Descicion Trees (``BDT'') and Deep Neural Networks (``DNN'') to tag boosted hadronically decaying $W$ bosons. Compared to simple 2-variable tagging approaches the multivariate approaches perform better, as can be seen in Figure \ref{fig:MVA}.



\begin{figure}[!h]
\begin{centering}
\includegraphics[width=0.48\textwidth]{WG3_plots/fig_07c}
\caption{Distributions showing comparison of the BDT and DNN taggers performance to a simple W tagger~\cite{ATL-PHYS-PUB-2017-004}.}
\label{fig:MVA}
\end{centering}
\end{figure}



\subsection{Jet substructure techniques for VBS in CMS}
% A. Hinzmann

Jet identification techniques that make use of jet substructure information are important tools for the measurement of VBS.
A brief summary of the existing tools used by the CMS experiment and future prospects for the HL-LHC is given in the following.

To probe high WW, ZZ or WZ invariant masses, special jet identification techniques for W and Z bosons decaying to quarks are needed, since for high momentum W and Z bosons, the shower of hadrons originating from the quark anti-quark pair merges into a single large radius jet of particles~\cite{CMS-PAS-JME-16-003, CMS-PAS-JME-14-002, Khachatryan:2014vla}.
The maximum angular separation between the quark and anti-quark is given by $\Delta R_{q\bar{q}}=2 m / p_{T}$, where $m$ and $p_T$ are the mass and transverse momentum, respectively, of the W or Z boson.
For a W boson with $p_T=1$ TeV, an angular separation of  $\Delta R_{q\bar{q}}=0.2$ is expected, which is well below the typical jet size parameter of 0.4 used by CMS.
At even higher $p_T=3.5$ TeV, the angular distance $\Delta R_{q\bar{q}}=0.05$ between the decay products of a W boson is even smaller than the granularity of the hadron calorimeter of CMS with cell sizes of $\Delta \eta \times \Delta \phi = 0.087 \times 0.087$ in the barrel region of the detector.
CMS thus employs a particle-flow event algorithm~\cite{Sirunyan:2017ulk} to measure jet substructure, that reconstructs and identifies each individual particle with an optimized combination of information from the various elements of the CMS detector, benefiting from spatial and energy resolution of all sub-detectors.

\begin{figure}[htb]
\begin{center}
% \hspace{-2cm}
\includegraphics[width=.35\textwidth]{WG3_plots/CMS-PAS-JME-16-003_Figure_018-c}
\includegraphics[width=.42\textwidth]{WG3_plots/CMS-PAS-JME-16-003_Figure_014-b}
% \vspace*{-1em}
\end{center}
\caption{(Left) Softdrop jet mass of boosted W bosons in data and simulated samples of top pair production in the single lepton plus jets final state.
(right) Pileup jet MVA discriminator in data and simulation for jets with $|\eta|>3.0$.}
\label{fig:CMSsubstructure}
\end{figure}

Figure~\ref{fig:CMSsubstructure} (left) shows the main observable used by CMS to distinguish W and Z boson jets from quark and gluon initiated jets, the softdrop jet mass, which is the mass of the jet after iteratively removing soft radiation with the modified mass-drop algorithm~\cite{Dasgupta:2013ihk,Butterworth:2008iy}, known as the softdrop algorithm~\cite{Larkoski:2014wba}, a procedure that reduces the mass of quark and gluon jets and improves the mass resolution of W and Z boson jets.
In addition, the substructure of the jet is explored with an N-subjettiness~\cite{Thaler:2010tr} ratio that distinguishes the W and Z boson jets composed of two hard subjets from single quark and gluon jets.
With this combination of observables, a mistag rate of $\sim1$\% at an efficiency of 50\% is achieved for a broad range of transverse momenta from $p_T>200$ GeV up to at least 3.5 TeV.
The jet mass and substructure observables are calibrated using a data sample of top pair production in the single lepton final state containing high-$p_T$ W bosons, achieving uncertainties of the order of 1\% on jet mass scale and 10\% on jet mass resolution and jet substructure tagging efficiency, which increase at higher jet $p_T$ where simulation is used for extrapolation.
Particles from additional interactions happening in the same pp bunch crossing, called pileup interactions, can significantly distort these observables.
CMS thus employs dedicated particle based pileup removal techniques that correct not only jet momenta, but also jet shape or substructure observables.
Charged particles that are identified by the tracking detector to originate from pileup interaction vertices are removed before jet clustering (this procedure is called CHS). For neutral particles a probability weight based on the distribution of surrounding particles following the pileup per particle identification (PUPPI) algorithm~\cite{Bertolini:2014bba} is applied to the particle four momenta~\cite{CMS-PAS-JME-14-001}.
With these pileup suppression techniques, the performance of W and Z boson identification is constant up to at least 40 pileup interactions, and with the higher granularity tracking detector planned for the HL-LHC, performance is maintained up to 200 pileup interactions.
Notable for VBS studies, the longitudinal and transverse polarization of W boson jets can be separated using subjet information, yielding a mistag rate of $\sim$30\% at an efficiency of 50\%.

Also the two forward jets from VBS require an analysis of their substructure, in order to suppress the background from (possibly overlapping) jets originating from pileup interactions~\cite{CMS-PAS-JME-16-003, CMS-PAS-JME-13-005}.
In a scenario of 25 pileup interactions (about half of the pileup expected in Run II of the LHC), without pileup mitigation $\sim$50\% of VBS selected forward jets are from pileup.
Since for $|\eta|>3.0$ no tracking information is available to suppress pileup particles and also jet shape difference are difficult to resolve due to coarse calorimeter granularity ($\Delta \eta \times \Delta \phi = 0.175 \times 0.175$), a multivariate analysis (MVA) is needed to distinguish quark jets from pileup and gluon background.
Figure~\ref{fig:CMSsubstructure} (right) shows the pileup jet MVA discriminator that allows to achieve a pileup mistag rate of $\sim$30\% at a quark efficiency of 50\%.
At the HL-LHC, for which CMS tracking and vertex identification will be extended from $|\eta|<2.5$ to $|\eta|<4$ and a high granularity endcap calorimeter will be installed in the region $1.5<|\eta|<3$, VBS jet identification can rely on the much more powerful CHS and PUPPI pileup rejection techniques.


\section*{Conclusions}

The VBSCan COST Action
gathers experts from the experimental, theoretical and statistics communities
to find a common strategy to tackle the great issues presented
by the measurement of the VBS processes at hadron colliders.
With an engagement that will last four years,
the community aims at going beyond the state of the art
in the instruments and knowledge needed for the best exploitation
of existing and future data 
that will be made available by the LHC and future colliders,
to determine the high-energy behaviour of the very core of the Standard Model,
the electroweak symmetry breaking.

\section*{Acknowledgements}

The author(s) would like to acknowledge the contribution of the COST Action CA16108.

%\bibliographystyle{plain}
%\bibliography{review} 

\printbibliography

\end{document}
